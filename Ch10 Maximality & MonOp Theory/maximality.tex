\documentclass[10pt,mathserif]{beamer}

\usepackage{graphicx,amsmath,amssymb,psfrag,mathtools}
\usepackage{soul}
\usepackage{amsmath,amsfonts,amsthm,bbm}
\usepackage{stmaryrd}
\usepackage{subcaption}



%Code block environment
\usepackage{listings}


\definecolor{lightgrey}{gray}{0.8}
\definecolor{medgrey}{gray}{0.6}
\definecolor{darkgrey}{gray}{0.4}
\usepackage{xcolor}
\lstset { %
    backgroundcolor=\color{black!5}, % set backgroundcolor
    basicstyle=\ttfamily,
    showstringspaces=false,
    commentstyle = \ttfamily,
    commentstyle=\color{commentgreen}\ttfamily,
    morecomment=[l][\color{darkgrey}]{//},
}



\usepackage{tikz}
\usetikzlibrary{matrix,chains,positioning,decorations.pathreplacing,arrows}
\usetikzlibrary{positioning,calc}
\usepackage{tkz-euclide}
%\usetkzobj{all}





%-------------------------------------------------------------------------------
%Definition of operator font
 \usepackage[bb=boondox]{mathalfa}
%% import \varmathbb without affecting other fonts
\usepackage{xparse}
\DeclareFontFamily{U}{ntxmia}{}
\DeclareFontShape{U}{ntxmia}{m}{it}{<-> ntxmia }{}
\DeclareFontShape{U}{ntxmia}{b}{it}{<-> ntxbmia }{}
\DeclareSymbolFont{lettersA}{U}{ntxmia}{m}{it}
\SetSymbolFont{lettersA}{bold}{U}{ntxmia}{b}{it}
\ExplSyntaxOn
\NewDocumentCommand{\varmathbb}{m}
 {
  \tl_map_inline:nn { #1 }
   {
    \use:c { varbb##1 }
   }
 }
\tl_map_inline:nn { ABCDEFGHIJKLMNOPQRSTUVWXYZ }
 {
  \exp_args:Nc \DeclareMathSymbol{varbb#1}{\mathord}{lettersA}{\int_eval:n { `#1+67 }}
 }
\exp_args:Nc \DeclareMathSymbol{varbbk}{\mathord}{lettersA}{169}
\ExplSyntaxOff
%%
\makeatletter
\DeclareFontFamily{U}{tipa}{}
\DeclareFontShape{U}{tipa}{m}{n}{<->tipa10}{}
\newcommand{\arc@char}{{\usefont{U}{tipa}{m}{n}\symbol{62}}}%

\newcommand{\arc}[1]{\mathpalette\arc@arc{#1}}

\newcommand{\arc@arc}[2]{%
  \sbox0{$\m@th#1#2$}%
  \vbox{
    \hbox{\resizebox{\wd0}{\height}{\arc@char}}
    \nointerlineskip
    \box0
  }%
}
\makeatother
\newcommand{\opA}{{\varmathbb{A}}}
\newcommand{\opB}{{\varmathbb{B}}}
\newcommand{\opC}{{\varmathbb{C}}}
\newcommand{\opD}{{\varmathbb{D}}}
\newcommand{\opE}{{\varmathbb{E}}}
\newcommand{\opF}{{\varmathbb{F}}}
\newcommand{\opG}{{\varmathbb{G}}}
\newcommand{\opH}{{\varmathbb{H}}}
\newcommand{\opI}{{\varmathbb{I}}}
\newcommand{\opJ}{{\varmathbb{J}}}
\newcommand{\opK}{{\varmathbb{K}}}
\newcommand{\opL}{{\varmathbb{L}}}
\newcommand{\opM}{{\varmathbb{M}}}
\newcommand{\opN}{{\varmathbb{N}}}
\newcommand{\opO}{{\varmathbb{O}}}
\newcommand{\opP}{{\varmathbb{P}}}
\newcommand{\opQ}{{\varmathbb{Q}}}
\newcommand{\opR}{{\varmathbb{R}}}
\newcommand{\opS}{{\varmathbb{S}}}
\newcommand{\opT}{{\varmathbb{T}}}
\newcommand{\opU}{{\varmathbb{U}}}
\newcommand{\opV}{{\varmathbb{V}}}
\newcommand{\opW}{{\varmathbb{W}}}
\newcommand{\opX}{{\varmathbb{X}}}
\newcommand{\opY}{{\varmathbb{Y}}}
\newcommand{\opZ}{{\varmathbb{Z}}}
\newcommand{\opZer}{\mathbb{0}}
%-------------------------------------------------------------------------------



%-------------------------------------------------------------------------------
%Definition of other font types
\newcommand{\va}{{\mathbf{a}}}
\newcommand{\vb}{{\mathbf{b}}}
\newcommand{\vc}{{\mathbf{c}}}
\newcommand{\vd}{{\mathbf{d}}}
\newcommand{\ve}{{\mathbf{e}}}
\newcommand{\vf}{{\mathbf{f}}}
\newcommand{\vg}{{\mathbf{g}}}
\newcommand{\vh}{{\mathbf{h}}}
\newcommand{\vi}{{\mathbf{i}}}
\newcommand{\vj}{{\mathbf{j}}}
\newcommand{\vk}{{\mathbf{k}}}
\newcommand{\vl}{{\mathbf{l}}}
\newcommand{\vm}{{\mathbf{m}}}
\newcommand{\vn}{{\mathbf{n}}}
\newcommand{\vo}{{\mathbf{o}}}
\newcommand{\vp}{{\mathbf{p}}}
\newcommand{\vq}{{\mathbf{q}}}
\newcommand{\vr}{{\mathbf{r}}}
\newcommand{\vs}{{\mathbf{s}}}
\newcommand{\vt}{{\mathbf{t}}}
\newcommand{\vu}{{\mathbf{u}}}
\newcommand{\vv}{{\mathbf{v}}}
\newcommand{\vw}{{\mathbf{w}}}
\newcommand{\vx}{{\mathbf{x}}}
\newcommand{\vy}{{\mathbf{y}}}
\newcommand{\vz}{{\mathbf{z}}}

\newcommand{\vA}{{\mathbf{A}}}
\newcommand{\vB}{{\mathbf{B}}}
\newcommand{\vC}{{\mathbf{C}}}
\newcommand{\vD}{{\mathbf{D}}}
\newcommand{\vE}{{\mathbf{E}}}
\newcommand{\vF}{{\mathbf{F}}}
\newcommand{\vG}{{\mathbf{G}}}
\newcommand{\vH}{{\mathbf{H}}}
\newcommand{\vI}{{\mathbf{I}}}
\newcommand{\vJ}{{\mathbf{J}}}
\newcommand{\vK}{{\mathbf{K}}}
\newcommand{\vL}{{\mathbf{L}}}
\newcommand{\vM}{{\mathbf{M}}}
\newcommand{\vN}{{\mathbf{N}}}
\newcommand{\vO}{{\mathbf{O}}}
\newcommand{\vP}{{\mathbf{P}}}
\newcommand{\vQ}{{\mathbf{Q}}}
\newcommand{\vR}{{\mathbf{R}}}
\newcommand{\vS}{{\mathbf{S}}}
\newcommand{\vT}{{\mathbf{T}}}
\newcommand{\vU}{{\mathbf{U}}}
\newcommand{\vV}{{\mathbf{V}}}
\newcommand{\vW}{{\mathbf{W}}}
\newcommand{\vX}{{\mathbf{X}}}
\newcommand{\vY}{{\mathbf{Y}}}
\newcommand{\vZ}{{\mathbf{Z}}}

\newcommand{\cA}{{\mathcal{A}}}
\newcommand{\cB}{{\mathcal{B}}}
\newcommand{\cC}{{\mathcal{C}}}
\newcommand{\cD}{{\mathcal{D}}}
\newcommand{\cE}{{\mathcal{E}}}
\newcommand{\cF}{{\mathcal{F}}}
\newcommand{\cG}{{\mathcal{G}}}
\newcommand{\cH}{{\mathcal{H}}}
\newcommand{\cI}{{\mathcal{I}}}
\newcommand{\cJ}{{\mathcal{J}}}
\newcommand{\cK}{{\mathcal{K}}}
\newcommand{\cL}{{\mathcal{L}}}
\newcommand{\cM}{{\mathcal{M}}}
\newcommand{\cN}{{\mathcal{N}}}
\newcommand{\cO}{{\mathcal{O}}}
\newcommand{\cP}{{\mathcal{P}}}
\newcommand{\cQ}{{\mathcal{Q}}}
\newcommand{\cR}{{\mathcal{R}}}
\newcommand{\cS}{{\mathcal{S}}}
\newcommand{\cT}{{\mathcal{T}}}
\newcommand{\cU}{{\mathcal{U}}}
\newcommand{\cV}{{\mathcal{V}}}
\newcommand{\cW}{{\mathcal{W}}}
\newcommand{\cX}{{\mathcal{X}}}
\newcommand{\cY}{{\mathcal{Y}}}
\newcommand{\cZ}{{\mathcal{Z}}}
%-------------------------------------------------------------------------------




%-------------------------------------------------------------------------------
%% macros for math notions and operators
\newcommand{\EE}{{\mathbb{E}}}
\newcommand{\expec}{\mathbb{E}}
\newcommand{\Prob}{{\mathrm{Prob}}} % probability

\newcommand{\reals}{\mathbb{R}}
\newcommand{\RR}{\mathbb{R}}
\newcommand{\complex}{\mathbb{C}}
\newcommand{\CC}{\mathbb{C}}
\newcommand{\nats}{\mathbb{N}}
\newcommand{\NN}{\mathbb{N}}
\newcommand{\ZZ}{\mathbb{Z}}
\newcommand{\bigO}{\mathcal{O}}
\newcommand{\order}[1]{{\mathcal{O}\left(#1\right)}}
\renewcommand{\SS}{{\mathbb{S}}}
\newcommand{\SSp}{\mathbb{S}_{+}}
\newcommand{\SSpp}{\mathbb{S}_{++}}
\newcommand{\sign}{\mathrm{sign}}
\newcommand{\vzero}{\mathbf{0}}
\newcommand{\vone}{{\mathbf{1}}}

\renewcommand{\Re}{\operatorname{Re}} 	%Real part
\renewcommand{\Im}{\operatorname{Im}}	%imaginary part

%\newcommand{\supp}{{\mathrm{supp}}} % support
\newcommand{\range}{\mathrm{range}\,} % domain
\newcommand{\tr}{{\mathrm{tr}}} % trace
%-------------------------------------------------------------------------------





%-------------------------------------------------------------------------------
%% Theorem definitions
\setbeamertemplate{theorems}[ams style] 
\newtheorem*{theorem*}{Theorem}
%\newtheorem{lemma}{Lemma}    % already provided by amsthm
\newtheorem{proposition}{Proposition}
%\newtheorem{proof}{Proof}  % already provided by amsthm


%-------------------------------------------------------------------------------
%% operator and convex analysis definitions

\newcommand*{\fix}{\mathrm{Fix}\,}
\newcommand*{\zer}{\mathrm{Zer}\,}
\newcommand*{\gra}{\mathrm{Gra}\,}
\newcommand{\prox}{\mathrm{Prox}}
\newcommand{\proj}{\Pi}
\newcommand{\aff}{\mathrm{aff}\,}    %affine hull
\newcommand{\intr}{\mathrm{int}\,}   %interior
\newcommand{\relint}{\mathrm{ri}\,}  %relative interior
\newcommand{\dom}{\mathrm{dom}\,} % domain
\newcommand{\epi}{\mathrm{epi}\,} % epigraph
\newcommand{\dist}{\mathrm{dist}}
\newcommand{\lagrange}{\mathbf{L}}  %saddle function
\newcommand{\fitzpatrick}{\mathbf{F}}   %Fitzpatrick function
\newcommand{\vecdelay}{\boldsymbol{d}}   %vector delay
\DeclareMathOperator*{\argmin}{argmin}
\DeclareMathOperator*{\argmax}{argmax}

%-------------------------------------------------------------------------------
%SRG definitions
\newcommand{\ereal}{\overline{\mathbb{R}^2}}
\newcommand{\ecomplex}{\overline{\mathbb{C}}}
\newcommand{\binfty}{{\boldsymbol \infty}}
\newcommand{\rarc}{\mathrm{Arc}^+}
\newcommand{\larc}{\mathrm{Arc}^-}


%-------------------------------------------------------------------------------




%-------------------------------------------------------------------------------
%Miscellaneous Stuff
%% sequences
\newcommand{\itom}{_{i=1}^{m}}
\newcommand{\ieqm}{i=1,\dots,m}

% use \numberthis to add an equation number in align*
\newcommand\numberthis{\addtocounter{equation}{1}\tag{\theequation}}

\newcolumntype{P}[1]{>{\centering\arraybackslash}p{#1}}

\mode<presentation>
{
\usetheme{default}
}
\setbeamertemplate{navigation symbols}{}
\usecolortheme[rgb={0.13,0.28,0.59}]{structure}
\setbeamertemplate{itemize subitem}{--}
\setbeamertemplate{frametitle} {
	\begin{center}
	  {\large\bf \insertframetitle}
	\end{center}
}

\newcommand\footlineon{
  \setbeamertemplate{footline} {
    \begin{beamercolorbox}[ht=2.5ex,dp=1.125ex,leftskip=.8cm,rightskip=.6cm]{structure}
      \footnotesize \insertsection
      \hfill
      {\insertframenumber}
    \end{beamercolorbox}
    \vskip 0.45cm
  }
}
\footlineon


\newcommand\footlineoff{
  \setbeamertemplate{footline} {
    \begin{beamercolorbox}[ht=2.5ex,dp=1.125ex,leftskip=.8cm,rightskip=.6cm]{structure}
      \footnotesize 
      \hfill
      {\insertframenumber}
    \end{beamercolorbox}
    \vskip 0.45cm
  }
}


\newcommand\blfootnote[1]{%
  \begingroup
  \renewcommand\thefootnote{}\footnote{#1}%
  \addtocounter{footnote}{-1}%
  \endgroup
}


\AtBeginSection[] 
{ 
	\begin{frame}<beamer> 
		\frametitle{Outline} 
		\tableofcontents[currentsection,currentsubsection] 
	\end{frame} 
} 

%% wotao's preference

        % itemize, black bullet, %150 spacing between items using "witemize"
        \newenvironment{witemize}{\itemize\addtolength{\itemsep}{0.3\baselineskip}}{\enditemize}

\iffalse
        % \setbeamertemplate{itemize items}[square]
        \setbeamertemplate{itemize items}{\textbullet}
        \setbeamercolor{itemize item}{fg=black}
        \setbeamercolor{itemize subitem}{fg=black}
        \setbeamercolor{itemize subsubitem}{fg=black}
        \setbeamercolor{enumerate item}{fg=black}
        \setbeamercolor{enumerate subitem}{fg=black}
        \setbeamercolor{enumerate subsubitem}{fg=black}
        \setbeamertemplate{itemize/enumerate body begin}{\normalsize}
        \setbeamertemplate{itemize/enumerate subbody begin}{\normalsize}
        \setbeamertemplate{itemize/enumerate subsubbody begin}{\normalsize}

        % itemize enumerate use normal sized texts
        \setbeamertemplate{itemize/enumerate body begin}{\normalsize}
        \setbeamertemplate{itemize/enumerate subbody begin}{\normalsize}
        \setbeamertemplate{itemize/enumerate subsubbody begin}{\normalsize}

        % block, black over gray with no shadow
        \setbeamertemplate{blocks}[rounded][shadow=false]
        \setbeamercolor{block title}{fg=black,bg=gray!40}
        \setbeamercolor{block body}{fg=black,bg=gray!10}
\fi

\author{Ernest K. Ryu and Wotao Yin}

\date{Large-Scale Convex Optimization via Monotone Operators}


\title{\large \bfseries Maximality and Monotone Operator Theory}

\begin{document}

\frame{
\thispagestyle{empty}
\titlepage
}


\begin{frame}
\frametitle{Monotone operator theory}
Convex optimization theory, the main subject of study in this class, focuses on the derivation and analysis of convex optimization algorithms.
\vspace{0.2in}

Monotone operator theory views monotone operators as interesting objects in their own right and focuses on understanding them better.

\vspace{0.2in}

One goal of this section is to provide theoretical completeness and prove several results that were simply asserted in other sections.
Another goal is to provide a gentle exposure to the field of monotone operator theory.

\vspace{0.2in}

Monotone operator theory takes place in infinite-dimensional Banach or Hilbert spaces, where a new set of interesting challenges arise.
We limit our attention to finite-dimensional Euclidean spaces.

\end{frame}


\section{Maximality of subdifferential}

\begin{frame}
\frametitle{Operator extensions}
$\bar{\opA}\colon\reals^n\rightrightarrows \reals^n$ is an extension of $\opA\colon\reals^n\rightrightarrows \reals^n$ if $\gra \bar{\opA} \supseteq \gra \opA $.\\
$\bar{\opA}$ is a proper extension of $\opA$ if the containment $\gra \bar{\opA} \supset \gra \opA$ is strict.
Recall, a monotone operator is maximal if it has no proper monotone extension.
\vspace{0.2in}



As discussed in \S2, if $\opA\colon\reals^n\rightrightarrows \reals^n$ is maximal monotone, then $\dom \opJ_\opA=\reals^n$.


\end{frame}


\begin{frame}
\frametitle{Maximality of subdifferential}
\setcounter{theorem}{6}
\begin{theorem}
\label{thm:subdiff-maximal}
If $f\colon\reals^n\rightarrow \reals^n\cup\{\infty\}$ is CCP, then $\partial f$ is maximal monotone.
\end{theorem}
\textbf{Proof.}
We know  $\partial f$ is monotone.
% By definition, $\partial f$ is maximal if and only if
% $(\tilde{x},\tilde{g})\notin \partial f$ implies 
% $\{(\tilde{x},\tilde{g})\}\cup \partial f$
% is not monotone.
% In other words, if $(\tilde{x},\tilde{g})\notin \partial f$, then
% there is $(x,g)\in \partial f$ such that
% $\langle g-\tilde g,x-\tilde x\rangle<0$.
Assume for contradiction that there is 
$(\tilde{x},\tilde{g})\notin \partial f$
such that
$\{(\tilde{x},\tilde{g})\}\cup \partial f$ is
monotone.
Define $(x,g)\in \partial f$ with
\[
x=\argmin_z\left\{f(z)+\frac{1}{2}\|z-(\tilde x+\tilde g)\|^2\right\},
\qquad
0= x-\tilde x+g-\tilde g.
\]
%We get $g\in \partial f(x)$ from the $0\in x-\tilde{x}+\partial f(x)-\tilde{g}$, the optimality condition of the $\argmin$.
%Since a strongly convex CCP function has a unique minimizer, $x$ is well defined.
Since $(\tilde x,\tilde g)\notin \partial f$, either
$x\ne \tilde x$ or $g\ne \tilde g$ (or both).
Using $x-\tilde{x}=-g+\tilde{g}$, we have
\[
\langle g-\tilde g,x-\tilde x\rangle=-\|x-\tilde x\|^2_2=-\|g-\tilde g\|^2_2
<0,
\]
which contradicts the assumption that
$\{(\tilde{x},\tilde{g})\}\cup \partial f$ is
monotone.
\qed
\end{frame}


\begin{frame}
\frametitle{Maximality of subdifferential}
Key idea of proof:
Given $v\in \reals^n$,
\[
v\mapsto (\underbrace{\prox_f(v)}_{=x},\underbrace{v-\prox_f(v)}_{=g})\in \partial f
\]
provides a unique decomposition $v=x+g$ such that $(x,g)\in \partial f$.
\end{frame}






\section{Fitzpatrick function}

\begin{frame}
\frametitle{Fitzpatrick function}
For $\opA\colon\reals^n\rightrightarrows \reals^n$, Fitzpatrick function $\fitzpatrick_\opA\colon\reals^n\times\reals^n\rightarrow\reals\cup\{\infty\}$ is
\[
\fitzpatrick_\opA(x,u)=\langle x,u\rangle -\inf_{(y,v)\in \opA}\langle x-y,u-v\rangle
=\sup_{(y,v)\in \opA}\left\{\langle y,u\rangle+\langle x,v\rangle-\langle y,v\rangle\right\}.
\]
%When $\opA$ is maximal monotone, $\fitzpatrick_\opA$ has certain nice properties.
%which is useful when $\opA$ is maximal monotone.

%(When $A$ is not maximal monotone, $\fitzpatrick_A$ does not seem to be useful.).
\setcounter{theorem}{2}
\begin{lemma}
\label{lem:fitzpatrick-properties}
Assume $\opA\colon\reals^n\rightrightarrows \reals^n$ is maximal monotone.
Then 
\begin{itemize}
    \item $\fitzpatrick_\opA$ is CCP,
    \item $\fitzpatrick_\opA(x,u)\ge \langle x,u\rangle$ for all $x,u\in \reals^n$, and
    \item $\fitzpatrick_\opA(x,u)=\langle x,u\rangle$ if and only if $(x,u)\in \opA$.
\end{itemize}
\end{lemma}
%The third point of Lemma~\ref{lem:fitzpatrick-properties} can be written as $\fitzpatrick_A(x,u)|_{\gra A}=\langle x,u\rangle$.
\end{frame}

\begin{frame}
\frametitle{Fitzpatrick function}
We say $\fitzpatrick_\opA$ is a representative function of $\opA$, since $\fitzpatrick_\opA$ is a convex extension of  $\langle x,u\rangle $ from $\gra \opA $ to $\reals^n\times \reals^n$ that furthermore satisfies $\fitzpatrick_\opA(x,u)\ge \langle x,u\rangle$.

%The Fitzpatrick function is one of the several representative functions used in the monotone operator theory literature.


\vspace{0.2in}
Common technique in monotone operator theory:
analyze a representative function to conclude results about the original operator.\\
Analyzing $\fitzpatrick_\opA$, a CCP function, with results from convex analysis is easier than directly considering $\opA$.
\end{frame}

\begin{frame}
\frametitle{Fitzpatrick function}
\textbf{Proof.}
If $(x,u)\in \opA$, then
$\langle x-y,u-v\rangle\ge 0$
for all $(y,v)\in \opA$ by monotonicity,
and the infimum
\[
\inf_{(y,v)\in \opA}\langle x-y,u-v\rangle=0
\]
is attained at $(x,u)$.
So $\fitzpatrick_\opA(x,u)=\langle x,u\rangle$.
\vspace{0.2in}

Assume $(x,u)\notin \opA$. Then by maximality there exists a $(y,v)\in \opA$ such that 
$\langle x-y,u-v\rangle<0$.
Therefore
\[
\inf_{(y,v)\in \opA}\langle x-y,u-v\rangle<0
\]
and $\fitzpatrick_\opA(x,u)>\langle x,u\rangle$.
\end{frame}

\begin{frame}
\frametitle{Fitzpatrick function}
Define
\[
f_{y,v}(x,u)=\langle y,u\rangle+\langle x,v\rangle-\langle y,v\rangle,
\]
which is a closed convex function for all $(y,v)\in \opA$.
Then
\vspace{-0.05in}

\[
\epi \fitzpatrick_\opA=\bigcap_{(y,v)\in \opA}\epi f_{y,v}
\]
\vspace{-0.1in}

is a closed convex set as it is an intersection of closed convex sets.
\vspace{0.2in}

Since $\fitzpatrick_\opA(x,u)\ge f_{y,v}(x,u)>-\infty$ for any $(y,v)\in \opA$,
we have $\fitzpatrick_\opA>-\infty$ always.
On the other hand, 
\[
\fitzpatrick_\opA(x,u)=\langle x,u\rangle <\infty
\]
for any $(x,u)\in \opA$. So $\fitzpatrick_\opA$ is proper.
\qed
\end{frame}



\begin{frame}
\frametitle{Minty surjectivity theorem}
The Minty surjectivity theorem is foundational to operator splitting methods as it ensures that methods using resolvents are well defined.

\vspace{0.2in}
We say $\opI+\opA$ is surjective if $\range (\opI+\opA)=\reals^n$.
%, i.e., for any $u\in \reals^n$ there is an $x\in \reals^n$ such that $u\in (\opI+\opA)x$.
If $\opI+\opA$ is surjective, then $\dom \opJ_\opA=\reals^n$.
\vspace{0.2in}

\setcounter{theorem}{7}
\begin{theorem}
%\emph{(Minty surjectivity theorem)}
\label{thm:minty}
If $\opA\colon\reals^n\rightrightarrows \reals^n$ is maximal monotone, then $\range(\opI+\opA)=\reals^n$.
\end{theorem}
\end{frame}


\begin{frame}
\frametitle{Minty surjectivity theorem}
\textbf{Proof.}
Want to show $u\in \range (\opI+\opA)$ for any  $u\in \reals^n$ and maximal monotone $\opA$.
We first establish $0\in \range(\opI+\opA)$ for any maximal monotone $\opA$.
Then the maximal monotone operator $\opB(x)=\opA(x)-u$ satisfies $0\in \range(\opI+\opB)$,
which implies $u\in \range(\opI+\opA)$ for any $u\in \reals^d$.



\end{frame}

\begin{frame}
\frametitle{Minty surjectivity theorem}
We now show $0\in \range(\opI+\opA)$.
Define $(y,v)\in \reals^n\times\reals^n$ with
\begin{equation*}
(y,v)=\argmin_{(x,u)\in\reals^n\times\reals^n}\left\{\fitzpatrick_\opA(x,u)+\frac{1}{2}\|x\|^2+\frac{1}{2}\|u\|^2\right\}=\prox_{\fitzpatrick_\opA}(0,0).
\label{mon-e0}
\end{equation*}
This implies 
\[
\begin{bmatrix}
-y\\-v
\end{bmatrix}\in \partial \fitzpatrick_\opA(y,v).
\]
Since $\fitzpatrick_\opA$ is convex, the subgradient inequality tells us
\[
\left<
\begin{bmatrix}
-y\\-v
\end{bmatrix},
\begin{bmatrix}
x\\u
\end{bmatrix}
-
\begin{bmatrix}
y\\v
\end{bmatrix}\right>\le \fitzpatrick_\opA(x,u)-\fitzpatrick_\opA(y,v)
\qquad
\forall\,(x,u)\in \reals^n\times\reals^n.
\]
By Lemma~\ref{lem:fitzpatrick-properties},
\[
\fitzpatrick_\opA(x,u)-\fitzpatrick_\opA(y,v)\le \langle x,u\rangle-\langle y,v\rangle
\qquad
\forall\,(x,u)\in \opA.
\]
\end{frame}

\begin{frame}
\frametitle{Minty surjectivity theorem}
Combining the two inequalities and reorganize to get
% \[
% \langle (-y,-v),(x,u)-(y,v)\rangle\le \langle x,u\rangle-\langle y,v\rangle,
% \]
% which we reorganize into
\begin{equation}
\|y+v\|^2\le \langle x+v,u+y\rangle\qquad\forall\, (x,u)\in \opA.
\label{mon-e1}
\end{equation}
Since $0\le \|y+v\|^2$ and since $\opA$ is maximal monotone, this implies $(-v,-y)\in \opA$.
By letting $(x,u)=(-v,-y)$ in \eqref{mon-e1}, we get $v=-y$.
Thus $(y,-y)\in \opA$ and we have
\[
0\in (\opI+\opA)(y).
\]
\qed
% Analogously note that \eqref{mon-e0} is  the crux of the proof.
% The point that establishes surjectivity
% is obtained as the minimizer of a strongly convex function, which exists.
\end{frame}





\begin{frame}
\frametitle{Converse of Minty}
The converse of Theorem~\ref{thm:minty} is true.
%Theorem~\ref{thm:minty-converse} 
As a consequence, we can show a monotone operator $\opA\colon\reals^n\rightrightarrows \reals^n$ is maximal if $\dom \opJ_\opA=\reals^n$.

\vspace{0.2in}

\begin{theorem}
\label{thm:minty-converse}
If $\opA\colon\reals^n\rightrightarrows \reals^n$ is monotone and $\range(J+\opA)=\reals^n$ for a symmetric positive definite $J\in\reals^{n\times n}$, then $\opA$ is maximal monotone.
\end{theorem}
\end{frame}


\begin{frame}[plain]
\frametitle{Converse of Minty}
\textbf{Proof.}
First consider the case $J=\opI$.
Assume $\{(x,u)\}\cup \opA$ is monotone, i.e.,
\[
0\le \langle x-z,u-w\rangle \qquad \forall(z,w)\in \opA.
\]
To establish maximality, enough to show $(x,u)\in \opA$.
Since $\range(\opI+\opA)=\reals^n$, there is a $y$ such that $x+u\in (\opI+\opA)y$.
Let
\[
v=x+u-y\in \opA y.
\]
Then
\[
0\le \langle x-y,u-v\rangle =-\|x-y\|^2=-\|u-v\|^2.
\]
So $x=y$ and $u=v$, which implies $(x,u)\in \opA$.

\vspace{0.2in}

When $J\ne \opI$. Then $J^{-1/2}\opA J^{-1/2}$ is monotone and, becuase $J+\opA$ is surjective,
\[
\range(\opI+J^{-1/2}\opA J^{-1/2})=\reals^n.
\]
This implies $J^{-1/2}\opA J^{-1/2}$ is maximal and so is $\opA$.
\qed
\end{frame}



\section{Maximality and extension theorems}
\begin{frame}
\frametitle{Maximality and extensions}
Let $P$ be a property of an operator such as monotonicity, $\theta$-averagedness, or $L$-Lipschitz continuity.
We say $\opA\colon\reals^n\rightrightarrows \reals^n$ is ``maximal $P$'' if there is no proper extension $\bar{\opA}$ with property $P$.
%To clarify, if $\opA$ is already maximal $P$, its maximal $P$ ``extension'' $\bar{\opA}$ is not proper, i.e., $\opA=\bar{\opA}$.
We now characterize maximal extensions of certain operator classes.

\vspace{0.2in}
Whether a given operator can be extended while preserving certain properties is a classical question in analysis.
(E.g., Hahn--Banach and Kirszbraun--Valentine theorems.)
\end{frame}


\begin{frame}
\frametitle{Maximal monotone extension}
\setcounter{theorem}{12}
\begin{theorem}
\label{thm:monotone-extension}
A monotone operator has a maximal monotone extension.
%Let $\opA\colon\reals^n\rightrightarrows \reals^n$ be monotone. Then $\opA$ has a maximal monotone extension $\bar{\opA}$.
%i.e., there exists a maximal monotone operator $\bar{\opA}\colon\reals^n\rightrightarrows \reals^n$ such that $\gra \opA\subseteq \gra \bar{\opA}$.
\end{theorem}
\textbf{Proof.}
Let $\opA\colon\reals^n\rightrightarrows \reals^n$ be monotone and let
\[
\cP=\{\opB\colon\reals^n\rightrightarrows \reals^n\,|\,\opB\text{ is monotone and } \gra\opA\subseteq\gra\opB\},
\]
which is non-empty. We impose the partial order on $\cP$ with $\opB_1\preceq \opB_2$ if and only if $\gra \opB_1\subseteq\gra\opB_2$ for all $\opB_1,\opB_2\in P$.
Every chain $\cC$ in $\cP$ has the upper bound $\bar{\opB}\in \cP$ given by
\[
\gra \bar{\opB}=\bigcup_{\opB \in \cC}\gra \opB.
\]
By Zorn's lemma, there is a maximal element $\bar{\opA}$ in $\cP$. This element $\bar{\opA}$ extends $\opA$ by definition of $\cP$ and cannot be properly extended as it is maximal in $\cP$.
\qed
\end{frame}

\begin{frame}[plain]
\frametitle{Maximal $\mu$-strongly monotone extension}
\begin{theorem}
\label{thm:sm-condition}
For $\mu>0$, a $\mu$-strongly monotone operator has a maximal $\mu$-strongly monotone extension.
Furthermore, if $\opA\colon\reals^n\rightrightarrows \reals^n$ is $\mu$-strongly monotone, then $\opA$ is maximal $\mu$-strongly monotone if and only if $\range( \opA)=\reals^n$.
% \[
% \langle \opA x-\opA y,x-y\rangle \ge \mu \|x-y\|^2
% \]
% and
\end{theorem}
\textbf{Proof.}
Since $\mu$-strong monotonicity of $\opA$ is defined as 
\[
\langle \opA x-\opA y,x-y\rangle \ge \mu \|x-y\|^2\qquad\forall\,x,y\in\reals^n,
\]
$\opA$ is $\mu$-strongly monotone if and only if $\opB=\opA-\mu\opI$ is monotone.

\vspace{0.05in}
Extending $\opA$ and $\opB$ are equivalent in the following sense.
If $\bar{\opA}$ is a $\mu$-strongly monotone extension of $\opA$, then $\bar{\opA}-\mu \opI$ is a monotone extension of $\opB$.
If $\bar{\opB}$ is a monotone extension of $\opB$, then $\bar{\opB}+\mu \opI$ is a $\mu$-strongly monotone extension of $\opA$.
By Theorem~\ref{thm:monotone-extension}, $\opB$ has a maximal monotone extension $\bar{\opB}$, and $\opA$ has a maximal $\mu$-strongly monotone extension $\bar{\opB}+\mu \opI$.

\vspace{0.05in}

Moreover, $\opA$ is maximal $\mu$-strongly monotone if and only if $\opB$ is maximal monotone.
By Theorems~\ref{thm:minty} and \ref{thm:minty-converse}, $\opB$ is maximal monotone if and only if $\range (\opA)=\range (\opB+\mu\opI)=\reals^n$.
Finally, chaining the equivalences provides the second stated result.
\qed
\end{frame}


\begin{frame}
\frametitle{Maximal $\beta$-cocoercive extension}
\begin{theorem}
\label{thm:coco-condition}
Let $\beta>0$.
A $\beta$-cocoercive operator has a maximal $\beta$-cocoercive extension.
Furthermore, if $\opA\colon\reals^n\rightrightarrows \reals^n$ is $\beta$-cocoercive, then $\opA$ is maximal $\beta$-cocoercive if and only if $\dom \opA=\reals^n$.
\end{theorem}
\textbf{Proof.}
Note $\opA$ is $\beta$-cocoercive if and only if $\opA^{-1}$ is $\beta$-strongly monotone.
\vspace{0.1in}

Extending $\opA$ and $\opA^{-1}$ are equivalent in the following sense.
If $\bar{\opA}$ is a $\beta$-cocoercive extension of $\opA$, then $\bar{\opA}^{-1}$ is a $\beta$-strongly monotone extension of $\opA^{-1}$.
If $\overline{\opA^{-1}}$ is a $\beta$-strongly monotone extension of $\opA^{-1}$, then $(\overline{\opA^{-1}})^{-1}$ is a $\beta$-cocoercive extension of $\opA$.
By Theorem~\ref{thm:sm-condition}, $\opA^{-1}$ has a maximal $\beta$-strongly monotone extension $\overline{\opA^{-1}}$, and $\opA$ has a maximal $\beta$-cocoercive extension $(\overline{\opA^{-1}})^{-1}$.

\end{frame}


\begin{frame}
\frametitle{Maximal $\beta$-cocoercive extension}
Moreover, $\opA$ is maximal $\beta$-cocoercive if and only if $\opA^{-1}$ is maximal $\beta$-strongly monotone.
By Theorem~\ref{thm:sm-condition}, $\opA^{-1}$ is maximal $\beta$-strongly monotone if and only if $\range (\opA^{-1})=\reals^n$, which holds if and only if $\dom (\opA)=\reals^n$.
Finally, chaining the equivalences provides the second stated result.
\qed
\vspace{0.2in}

Remember that a $\beta$-cocoercive operators must be single-valued.
By Theorem~\ref{thm:coco-condition}, [$\opA\colon\reals^n\rightrightarrows \reals^n$ is maximal $\beta$-cocoercive] is equivalent to [$\opA\colon\reals^n\rightarrow \reals^n$ is $\beta$-cocoercive] since $\opA\colon\reals^n\rightarrow \reals^n$ implies $\dom \opA=\reals^n$.
For the sake of conciseness, we usually avoid the former expression.
\end{frame}




\begin{frame}
\frametitle{Maximal $L$-Lipschitz extension}
\begin{theorem}
\label{thm:Lipschitz-condition}
Let $L>0$.
An $L$-Lipschitz operator has a maximal $L$-Lipschitz extension.
Furthermore, if $\opA\colon\reals^n\rightrightarrows \reals^n$ is $L$-Lipschitz, then $\opA$ is maximal $L$-Lipschitz if and only if $\dom \opA=\reals^n$.
\end{theorem}

\vspace{0.2in}
This result is known as the Kirszbraun--Valentine theorem. 
Proof follows from similar reasoning.
%We defer the proof to Exercise~\ref{exercise:maximal-nonexpansive}.

\end{frame}

\end{document}
