\documentclass[10pt,mathserif]{beamer}

\usepackage{graphicx,amsmath,amssymb,psfrag,mathtools}
\usepackage{soul}
\usepackage{amsmath,amsfonts,amsthm,bbm}
\usepackage{stmaryrd}
\usepackage{subcaption}



%Code block environment
\usepackage{listings}


\definecolor{lightgrey}{gray}{0.8}
\definecolor{medgrey}{gray}{0.6}
\definecolor{darkgrey}{gray}{0.4}
\usepackage{xcolor}
\lstset { %
    backgroundcolor=\color{black!5}, % set backgroundcolor
    basicstyle=\ttfamily,
    showstringspaces=false,
    commentstyle = \ttfamily,
    commentstyle=\color{commentgreen}\ttfamily,
    morecomment=[l][\color{darkgrey}]{//},
}



\usepackage{tikz}
\usetikzlibrary{matrix,chains,positioning,decorations.pathreplacing,arrows}
\usetikzlibrary{positioning,calc}
\usepackage{tkz-euclide}
%\usetkzobj{all}





%-------------------------------------------------------------------------------
%Definition of operator font
 \usepackage[bb=boondox]{mathalfa}
%% import \varmathbb without affecting other fonts
\usepackage{xparse}
\DeclareFontFamily{U}{ntxmia}{}
\DeclareFontShape{U}{ntxmia}{m}{it}{<-> ntxmia }{}
\DeclareFontShape{U}{ntxmia}{b}{it}{<-> ntxbmia }{}
\DeclareSymbolFont{lettersA}{U}{ntxmia}{m}{it}
\SetSymbolFont{lettersA}{bold}{U}{ntxmia}{b}{it}
\ExplSyntaxOn
\NewDocumentCommand{\varmathbb}{m}
 {
  \tl_map_inline:nn { #1 }
   {
    \use:c { varbb##1 }
   }
 }
\tl_map_inline:nn { ABCDEFGHIJKLMNOPQRSTUVWXYZ }
 {
  \exp_args:Nc \DeclareMathSymbol{varbb#1}{\mathord}{lettersA}{\int_eval:n { `#1+67 }}
 }
\exp_args:Nc \DeclareMathSymbol{varbbk}{\mathord}{lettersA}{169}
\ExplSyntaxOff
%%
\makeatletter
\DeclareFontFamily{U}{tipa}{}
\DeclareFontShape{U}{tipa}{m}{n}{<->tipa10}{}
\newcommand{\arc@char}{{\usefont{U}{tipa}{m}{n}\symbol{62}}}%

\newcommand{\arc}[1]{\mathpalette\arc@arc{#1}}

\newcommand{\arc@arc}[2]{%
  \sbox0{$\m@th#1#2$}%
  \vbox{
    \hbox{\resizebox{\wd0}{\height}{\arc@char}}
    \nointerlineskip
    \box0
  }%
}
\makeatother
\newcommand{\opA}{{\varmathbb{A}}}
\newcommand{\opB}{{\varmathbb{B}}}
\newcommand{\opC}{{\varmathbb{C}}}
\newcommand{\opD}{{\varmathbb{D}}}
\newcommand{\opE}{{\varmathbb{E}}}
\newcommand{\opF}{{\varmathbb{F}}}
\newcommand{\opG}{{\varmathbb{G}}}
\newcommand{\opH}{{\varmathbb{H}}}
\newcommand{\opI}{{\varmathbb{I}}}
\newcommand{\opJ}{{\varmathbb{J}}}
\newcommand{\opK}{{\varmathbb{K}}}
\newcommand{\opL}{{\varmathbb{L}}}
\newcommand{\opM}{{\varmathbb{M}}}
\newcommand{\opN}{{\varmathbb{N}}}
\newcommand{\opO}{{\varmathbb{O}}}
\newcommand{\opP}{{\varmathbb{P}}}
\newcommand{\opQ}{{\varmathbb{Q}}}
\newcommand{\opR}{{\varmathbb{R}}}
\newcommand{\opS}{{\varmathbb{S}}}
\newcommand{\opT}{{\varmathbb{T}}}
\newcommand{\opU}{{\varmathbb{U}}}
\newcommand{\opV}{{\varmathbb{V}}}
\newcommand{\opW}{{\varmathbb{W}}}
\newcommand{\opX}{{\varmathbb{X}}}
\newcommand{\opY}{{\varmathbb{Y}}}
\newcommand{\opZ}{{\varmathbb{Z}}}
\newcommand{\opZer}{\mathbb{0}}
%-------------------------------------------------------------------------------



%-------------------------------------------------------------------------------
%Definition of other font types
\newcommand{\va}{{\mathbf{a}}}
\newcommand{\vb}{{\mathbf{b}}}
\newcommand{\vc}{{\mathbf{c}}}
\newcommand{\vd}{{\mathbf{d}}}
\newcommand{\ve}{{\mathbf{e}}}
\newcommand{\vf}{{\mathbf{f}}}
\newcommand{\vg}{{\mathbf{g}}}
\newcommand{\vh}{{\mathbf{h}}}
\newcommand{\vi}{{\mathbf{i}}}
\newcommand{\vj}{{\mathbf{j}}}
\newcommand{\vk}{{\mathbf{k}}}
\newcommand{\vl}{{\mathbf{l}}}
\newcommand{\vm}{{\mathbf{m}}}
\newcommand{\vn}{{\mathbf{n}}}
\newcommand{\vo}{{\mathbf{o}}}
\newcommand{\vp}{{\mathbf{p}}}
\newcommand{\vq}{{\mathbf{q}}}
\newcommand{\vr}{{\mathbf{r}}}
\newcommand{\vs}{{\mathbf{s}}}
\newcommand{\vt}{{\mathbf{t}}}
\newcommand{\vu}{{\mathbf{u}}}
\newcommand{\vv}{{\mathbf{v}}}
\newcommand{\vw}{{\mathbf{w}}}
\newcommand{\vx}{{\mathbf{x}}}
\newcommand{\vy}{{\mathbf{y}}}
\newcommand{\vz}{{\mathbf{z}}}

\newcommand{\vA}{{\mathbf{A}}}
\newcommand{\vB}{{\mathbf{B}}}
\newcommand{\vC}{{\mathbf{C}}}
\newcommand{\vD}{{\mathbf{D}}}
\newcommand{\vE}{{\mathbf{E}}}
\newcommand{\vF}{{\mathbf{F}}}
\newcommand{\vG}{{\mathbf{G}}}
\newcommand{\vH}{{\mathbf{H}}}
\newcommand{\vI}{{\mathbf{I}}}
\newcommand{\vJ}{{\mathbf{J}}}
\newcommand{\vK}{{\mathbf{K}}}
\newcommand{\vL}{{\mathbf{L}}}
\newcommand{\vM}{{\mathbf{M}}}
\newcommand{\vN}{{\mathbf{N}}}
\newcommand{\vO}{{\mathbf{O}}}
\newcommand{\vP}{{\mathbf{P}}}
\newcommand{\vQ}{{\mathbf{Q}}}
\newcommand{\vR}{{\mathbf{R}}}
\newcommand{\vS}{{\mathbf{S}}}
\newcommand{\vT}{{\mathbf{T}}}
\newcommand{\vU}{{\mathbf{U}}}
\newcommand{\vV}{{\mathbf{V}}}
\newcommand{\vW}{{\mathbf{W}}}
\newcommand{\vX}{{\mathbf{X}}}
\newcommand{\vY}{{\mathbf{Y}}}
\newcommand{\vZ}{{\mathbf{Z}}}

\newcommand{\cA}{{\mathcal{A}}}
\newcommand{\cB}{{\mathcal{B}}}
\newcommand{\cC}{{\mathcal{C}}}
\newcommand{\cD}{{\mathcal{D}}}
\newcommand{\cE}{{\mathcal{E}}}
\newcommand{\cF}{{\mathcal{F}}}
\newcommand{\cG}{{\mathcal{G}}}
\newcommand{\cH}{{\mathcal{H}}}
\newcommand{\cI}{{\mathcal{I}}}
\newcommand{\cJ}{{\mathcal{J}}}
\newcommand{\cK}{{\mathcal{K}}}
\newcommand{\cL}{{\mathcal{L}}}
\newcommand{\cM}{{\mathcal{M}}}
\newcommand{\cN}{{\mathcal{N}}}
\newcommand{\cO}{{\mathcal{O}}}
\newcommand{\cP}{{\mathcal{P}}}
\newcommand{\cQ}{{\mathcal{Q}}}
\newcommand{\cR}{{\mathcal{R}}}
\newcommand{\cS}{{\mathcal{S}}}
\newcommand{\cT}{{\mathcal{T}}}
\newcommand{\cU}{{\mathcal{U}}}
\newcommand{\cV}{{\mathcal{V}}}
\newcommand{\cW}{{\mathcal{W}}}
\newcommand{\cX}{{\mathcal{X}}}
\newcommand{\cY}{{\mathcal{Y}}}
\newcommand{\cZ}{{\mathcal{Z}}}
%-------------------------------------------------------------------------------




%-------------------------------------------------------------------------------
%% macros for math notions and operators
\newcommand{\EE}{{\mathbb{E}}}
\newcommand{\expec}{\mathbb{E}}
\newcommand{\Prob}{{\mathrm{Prob}}} % probability

\newcommand{\reals}{\mathbb{R}}
\newcommand{\RR}{\mathbb{R}}
\newcommand{\complex}{\mathbb{C}}
\newcommand{\CC}{\mathbb{C}}
\newcommand{\nats}{\mathbb{N}}
\newcommand{\NN}{\mathbb{N}}
\newcommand{\ZZ}{\mathbb{Z}}
\newcommand{\bigO}{\mathcal{O}}
\newcommand{\order}[1]{{\mathcal{O}\left(#1\right)}}
\renewcommand{\SS}{{\mathbb{S}}}
\newcommand{\SSp}{\mathbb{S}_{+}}
\newcommand{\SSpp}{\mathbb{S}_{++}}
\newcommand{\sign}{\mathrm{sign}}
\newcommand{\vzero}{\mathbf{0}}
\newcommand{\vone}{{\mathbf{1}}}

\renewcommand{\Re}{\operatorname{Re}} 	%Real part
\renewcommand{\Im}{\operatorname{Im}}	%imaginary part

%\newcommand{\supp}{{\mathrm{supp}}} % support
\newcommand{\range}{\mathrm{range}\,} % domain
\newcommand{\tr}{{\mathrm{tr}}} % trace
%-------------------------------------------------------------------------------





%-------------------------------------------------------------------------------
%% Theorem definitions
\setbeamertemplate{theorems}[ams style] 
\newtheorem*{theorem*}{Theorem}
%\newtheorem{lemma}{Lemma}    % already provided by amsthm
\newtheorem{proposition}{Proposition}
%\newtheorem{proof}{Proof}  % already provided by amsthm


%-------------------------------------------------------------------------------
%% operator and convex analysis definitions

\newcommand*{\fix}{\mathrm{Fix}\,}
\newcommand*{\zer}{\mathrm{Zer}\,}
\newcommand*{\gra}{\mathrm{Gra}\,}
\newcommand{\prox}{\mathrm{Prox}}
\newcommand{\proj}{\Pi}
\newcommand{\aff}{\mathrm{aff}\,}    %affine hull
\newcommand{\intr}{\mathrm{int}\,}   %interior
\newcommand{\relint}{\mathrm{ri}\,}  %relative interior
\newcommand{\dom}{\mathrm{dom}\,} % domain
\newcommand{\epi}{\mathrm{epi}\,} % epigraph
\newcommand{\dist}{\mathrm{dist}}
\newcommand{\lagrange}{\mathbf{L}}  %saddle function
\newcommand{\fitzpatrick}{\mathbf{F}}   %Fitzpatrick function
\newcommand{\vecdelay}{\boldsymbol{d}}   %vector delay
\DeclareMathOperator*{\argmin}{argmin}
\DeclareMathOperator*{\argmax}{argmax}

%-------------------------------------------------------------------------------
%SRG definitions
\newcommand{\ereal}{\overline{\mathbb{R}^2}}
\newcommand{\ecomplex}{\overline{\mathbb{C}}}
\newcommand{\binfty}{{\boldsymbol \infty}}
\newcommand{\rarc}{\mathrm{Arc}^+}
\newcommand{\larc}{\mathrm{Arc}^-}


%-------------------------------------------------------------------------------




%-------------------------------------------------------------------------------
%Miscellaneous Stuff
%% sequences
\newcommand{\itom}{_{i=1}^{m}}
\newcommand{\ieqm}{i=1,\dots,m}

% use \numberthis to add an equation number in align*
\newcommand\numberthis{\addtocounter{equation}{1}\tag{\theequation}}

\newcolumntype{P}[1]{>{\centering\arraybackslash}p{#1}}

\mode<presentation>
{
\usetheme{default}
}
\setbeamertemplate{navigation symbols}{}
\usecolortheme[rgb={0.13,0.28,0.59}]{structure}
\setbeamertemplate{itemize subitem}{--}
\setbeamertemplate{frametitle} {
	\begin{center}
	  {\large\bf \insertframetitle}
	\end{center}
}

\newcommand\footlineon{
  \setbeamertemplate{footline} {
    \begin{beamercolorbox}[ht=2.5ex,dp=1.125ex,leftskip=.8cm,rightskip=.6cm]{structure}
      \footnotesize \insertsection
      \hfill
      {\insertframenumber}
    \end{beamercolorbox}
    \vskip 0.45cm
  }
}
\footlineon


\newcommand\footlineoff{
  \setbeamertemplate{footline} {
    \begin{beamercolorbox}[ht=2.5ex,dp=1.125ex,leftskip=.8cm,rightskip=.6cm]{structure}
      \footnotesize 
      \hfill
      {\insertframenumber}
    \end{beamercolorbox}
    \vskip 0.45cm
  }
}


\newcommand\blfootnote[1]{%
  \begingroup
  \renewcommand\thefootnote{}\footnote{#1}%
  \addtocounter{footnote}{-1}%
  \endgroup
}


\AtBeginSection[] 
{ 
	\begin{frame}<beamer> 
		\frametitle{Outline} 
		\tableofcontents[currentsection,currentsubsection] 
	\end{frame} 
} 

%% wotao's preference

        % itemize, black bullet, %150 spacing between items using "witemize"
        \newenvironment{witemize}{\itemize\addtolength{\itemsep}{0.3\baselineskip}}{\enditemize}

\iffalse
        % \setbeamertemplate{itemize items}[square]
        \setbeamertemplate{itemize items}{\textbullet}
        \setbeamercolor{itemize item}{fg=black}
        \setbeamercolor{itemize subitem}{fg=black}
        \setbeamercolor{itemize subsubitem}{fg=black}
        \setbeamercolor{enumerate item}{fg=black}
        \setbeamercolor{enumerate subitem}{fg=black}
        \setbeamercolor{enumerate subsubitem}{fg=black}
        \setbeamertemplate{itemize/enumerate body begin}{\normalsize}
        \setbeamertemplate{itemize/enumerate subbody begin}{\normalsize}
        \setbeamertemplate{itemize/enumerate subsubbody begin}{\normalsize}

        % itemize enumerate use normal sized texts
        \setbeamertemplate{itemize/enumerate body begin}{\normalsize}
        \setbeamertemplate{itemize/enumerate subbody begin}{\normalsize}
        \setbeamertemplate{itemize/enumerate subsubbody begin}{\normalsize}

        % block, black over gray with no shadow
        \setbeamertemplate{blocks}[rounded][shadow=false]
        \setbeamercolor{block title}{fg=black,bg=gray!40}
        \setbeamercolor{block body}{fg=black,bg=gray!10}
\fi

\author{Ernest K. Ryu and Wotao Yin}

\date{Large-Scale Convex Optimization via Monotone Operators}


\title{\large \bfseries Duality in Splitting Methods}

\begin{document}

\frame{
\thispagestyle{empty}
\titlepage
}


\begin{frame}
\frametitle{Attouch--Th\'era duality and splitting methods}

We present Attouch--Th\'era duality, which is analogous to, but \emph{simpler} than, convex duality,
and explore its connection to base splitting methods.
%This was an omitted theoretical detail.

%Furthermore, Attouch--Th\'era duality has algorithmic utility as dual solutions certify correctness of primal solutions.
%
%
%We now present the intimate connection of the base splittings of \S\ref{ss-base-splitting} with Attouch--Th\'era duality. We also show that the splittings are primal-dual in the sense that they provide dual information.

\end{frame}



\section{Fenchel duality}
\begin{frame}
\frametitle{Fenchel duality}
Fenchel duality: primal
\[
\begin{array}{ll}
\underset{x\in\reals^n}{\mathrm{minimize}} &
f(x)+g(x),
\end{array}
\label{eq:fenchelP}
%\tag{F-P}
\]
%where $f$ and $g$ are CCP functions,
and dual
\[\begin{array}{ll}
\underset{u \in\reals^n}{\mathrm{maximize}} &
-f^*(-u )-g^*(u )
\end{array}
\label{eq:fenchelD}
%\tag{F-D}
\]
generated by 
\[
\lagrange(x,u )=f(x)+\langle x,u \rangle-g^*(u ).
\]
\vspace{0.2in}

Total duality is subtle.
\end{frame}

\begin{frame}
\frametitle{One Interpretation of Fenchel duality}
%We now discuss an interpretation of Fenchel duality that we later extend to operators.
For simplicity, assume total duality and $f$, $g$, $f^*$, and $g^*$ differentiable.
\vspace{0.2in}


Primal is to find point $x$ such that $\nabla f$ and $\nabla g$ at $x$ sum to $0$:
\[ 
\begin{array}{ll}
\underset{x\in\reals^n}{\mathrm{find}} &
0=\nabla f(x)+\nabla g(x),
\end{array}
\]
\vspace{0.2in}

Dual is to find gradient $u $ such that $\nabla f$ produces $-u $ and $\nabla g$ produces $u $ at the same point:
\[ 
\begin{array}{ll}
\underset{u \in\reals^n}{\mathrm{find}} &
(\nabla f)^{-1}(-u )=(\nabla g)^{-1}(u )
\end{array}
\]


\vspace{0.2in}
This is one of the many viewpoints of convex duality.

%Primal viewpoint is to find point $x$. Dual viewpoint  is to find gradient $u $.
\end{frame}







\section{Attouch--Th\'era duality}
\begin{frame}
\frametitle{Attouch--Th\'era duality}
Consider
\[ 
\begin{array}{ll}
\underset{x\in\reals^n}{\mathrm{find}} &
0\in (\opA+\opB)x,
\end{array}
\]
where $\opA$ and $\opB$ are maximal monotone.

\vspace{0.2in}

Define $\opA^{-\ovee}(u )=-\opA^{-1}(-u )$.
\vspace{0.2in}

 Attouch--Th\'era dual monotone inclusion problem is 
\[ 
\begin{array}{ll}
\underset{u \in\reals^n}{\mathrm{find}} &
0\in (\opA^{-\ovee}+\opB^{-1})u .
\end{array}
\]
\end{frame}

\begin{frame}
\frametitle{Attouch--Th\'era duality}
Attouch--Th\'era duality is, in a sense, easier than Fenchel duality since
\[
\zer(\opA+\opB)\ne\emptyset
\quad\Leftrightarrow\quad
\zer(\opA^{-\ovee}+\opB^{-1})\ne\emptyset,
\]
i.e., a primal solution exists if and only if a dual solution exists.
\vspace{0.2in}

\textbf{Proof.}
\begin{align*}
\exists\,x
\left[0\in (\opA +\opB )x\right]
%&\quad\Leftrightarrow\quad
%\exists\,x,u 
%\left[0\in (\opA +\opB )x,\,
%-u \in \opA x,\,u  \in \opB x\right]\\
\quad\Leftrightarrow\quad&
\exists\,x,u 
\left[-u \in \opA x,\,u  \in \opB x\right]\\
\quad\Leftrightarrow\quad&
\exists\,x,u 
\left[-x\in \opA ^{-\ovee}u ,\,x \in \opB ^{-1}u \right]\\
%\quad\Leftrightarrow\quad&
%\exists\,x,u 
%\left[0\in (\opA ^{-\ovee}+\opB ^{-1})u ,\,
%-x\in \opA ^{-\ovee}u ,\,x \in \opB ^{-1}u \right]\\
\quad\Leftrightarrow\quad&
\exists\,u 
\left[0\in (\opA ^{-\ovee}+\opB ^{-1})u \right].
\end{align*}
\qed

%In Attouch--Th\'era duality, 
(No notion of strong duality, since no function values.)
\end{frame}

\begin{frame}
\frametitle{Attouch--Th\'era vs.\ Fenchel duality}

Attouch--Th\'era generalizes Fenchel duality in the following sense:
\[
\partial\text{(proper convex function)}\subset \text{monotone operators}
\]
%as monotone operators generalize subdifferential operators of convex functions.

\vspace{0.2in}
However, Attouch--Th\'era fails to capture the subtleties of Fenchel duality.

\vspace{0.2in}


In Fenchel duality, strong duality may fail, a primal solution may exist while a dual solution does not, or vice versa.
No analogous pathologies in Attouch--Th\'era duality.
\end{frame}


\begin{frame}
\frametitle{Dual solutions as certificates}
It is desirable for a method to produce both primal and dual solutions as the dual solution can certify correctness of the primal solution.

\vspace{0.2in}

If a primal-dual solution $(x^\star,u ^\star)$ satisfying 
\begin{equation}
-u ^\star\in \opA x^\star\text{ and }u ^\star\in \opB x^\star
\label{eq:primal-dual-inclusion-sol}
\end{equation}
 is provided, verifying \eqref{eq:primal-dual-inclusion-sol} certifies correctness of the solutions.
%A dual solution can certify correctness of a primal solution. 



%(This check also verifies that $u ^\star$ is indeed a dual solution.)
\vspace{0.2in}


If only a primal solution $x^\star$ is provided, we must verify $0\in \opA x^\star+\opB x^\star$. How do we compute the 
Minkowski sum $\opA x^\star+\opB x^\star$?

% This can be difficult if there is no effective way to compute the 
%(On a computer, how would we represent the sets $Ax^\star$ and $Bx^\star$ and how would we compute the Minkowski sum?)
%Therefore, it is desirable for a splitting method to produce solutions of both the primal and dual monotone inclusion problems.
%i.e., we want our methods to be primal-dual.

%In practice, a method used to verify ``correctness'' of a primal-dual solution must be able to deal with inaccuracies, since an output an iterative algorithm will be at most approximately correct. This issue relates how to design an effective termination criterion for the iterative methods. We avoid this discussion for the sake of simplicity.

%For example, we will soon talk about how DRS produces the iterates $(x^{k+1/2},u ^{k+1/2})\in B$ and $(x^{k+1},-u ^{k+1})\in A$ such that $x^{k+1}-x^{k+1/2}\rightarrow 0$ and $u ^{k+1}-u ^{k+1/2}\rightarrow 0$.
\end{frame}



\section{Duality in splitting methods}

\begin{frame}
\frametitle{FBS}
The FPI with FBS
\begin{align*}
x^{k+1/2}&=x^k-\alpha \opA x^k\\
x^{k+1}&=\opJ_{\alpha \opB}x^{k+1/2}
\end{align*}
 often not considered a primal-dual method.
We can make it primal-dual:
\begin{align*}
x^{k+1/2}&=x^k-\alpha \opA x^k\\
u ^{k+1/2}&=-\opA x^k\\
x^{k+1}&=\opJ_{\alpha \opB}x^{k+1/2}\\
u ^{k+1}&=\alpha^{-1}(x^{k+1/2}-x^{k+1}).
\end{align*}
Note $u ^{k+1}\in \opB x^{k+1}$.
If $x^k\rightarrow x^\star$, then
\[
u ^{k+1/2}\rightarrow u ^\star
,\quad
u ^{k+1}\rightarrow u ^\star,\quad
u ^\star\in \zer(\opA^{-\ovee}+\opB^{-1}).
\]
\end{frame}


\begin{frame}{Characterization of fixed points of DRS}
With Attouch--Th\'era dual, characterize fixed points of  PRS and DRS:
\[
\fix\left(\opR_{\alpha \opA}\opR_{\alpha \opB}\right)\subseteq\zer(\opA+\opB)+\alpha \zer(\opA^{-\ovee}+\opB^{-1})
\]
\textbf{Proof.}
%$x$ is a fixed point of the PRS operator if and only if
\begin{align*}
z=&\opR_{\alpha \opA}\opR_{\alpha \opB}z\\
&\quad\Leftrightarrow\quad z+2\opJ_{\alpha \opA  }(2\opJ_{\alpha \opB }-\opI)z-2\opJ_{\alpha \opB }z=z,\,x=\opJ_{\alpha \opB }z\\
%&\quad\Leftrightarrow\quad\opJ_{\alpha \opA  }(2x-z)=x,\,z=x+\alpha u ,\,u \in \opB x\\
&\quad\Leftrightarrow\quad
\opJ_{\alpha \opA  }(
x-\alpha u  
)=x,\,z=x+\alpha u ,\,u \in \opB x\\
&\quad\Leftrightarrow\quad
x-\alpha u 
=x+\alpha  v ,\, v \in \opA  x,\,z=x+\alpha u ,\,u \in \opB x\\
&\quad\Leftrightarrow\quad
 v =- u ,\, v \in \opA  x,\,
 u \in \opB x,\,z=x+\alpha  u \\
&\quad\Leftrightarrow\quad- u \in \opA  x,\, u \in \opB x,\,z=x+\alpha  u \\
&\quad\Leftrightarrow\quad- u \in \opA  x,\, u \in \opB x
,\,
-x\in \opA  ^{-\ovee} u ,\,x\in \opB ^{-1} u 
,\,z=x+\alpha  u \\
&\quad\Rightarrow\quad
0\in (\opA  +\opB )x,\,
0\in (\opA  ^{-\ovee}+\opB ^{-1}) u ,\,z=x+\alpha  u .
\end{align*}
Last step is not an equivalence, so 
characterization with $\subseteq$, not $=$.
\end{frame}

\begin{frame}
\frametitle{Primal-dual DRS}
We can make the FPI with DRS more explicitly primal-dual:
\begin{align*}
x^{k+1/2}&=\opJ_{\alpha \opB }(z^k)\\
 u ^{k+1/2}&=\frac{1}{\alpha}(z^k-x^{k+1/2})\\
x^{k+1}&=\opJ_{\alpha \opA }(2x^{k+1/2}-z^k)\\
 u ^{k+1}&=\frac{1}{\alpha}(x^{k+1}-x^{k+1/2}+\alpha u ^{k+1/2})\\
z^{k+1}&=z^k+x^{k+1}-x^{k+1/2}.
\end{align*}
Note $ u ^{k+1/2}\in \opB x^{k+1/2}$, $- u ^{k+1}\in \opA x^{k+1}$.
If $\zer(\opA +\opB )\ne\emptyset$, then
\begin{gather*}
x^{k+1/2}\rightarrow x^\star,\quad x^{k+1}\rightarrow x^\star,\quad
x^\star\in \zer(\opA +\opB )\\
 u ^{k+1/2}\rightarrow  u ^\star
,\quad
 u ^{k+1}\rightarrow  u ^\star,\quad
 u ^\star\in \zer(\opA ^{-\ovee}+\opB ^{-1})\\
z^k\rightarrow x^\star+\alpha u ^\star.
\end{gather*}
\end{frame}

\begin{frame}
\frametitle{Self-dual property of DRS}
PRS and DRS are self-dual:
\[
%(2\opJ_{\opA }-\opI )(2\opJ_\opB -\opI )=(2\opJ_{\opA ^{-\ovee}}-\opI )(2\opJ_{\opB ^{-1}}-\opI ).
\opR_{\opA }\opR_\opB=\opR_{\opA ^{-\ovee}}\opR_{\opB ^{-1}}
\]

\vspace{0.2in}

Follows from $\opJ_{\opA ^{-\ovee}}=\opI +\opJ_{\opA }(-\opI )$
and $\opJ_{\opB ^{-1}}=\opI -\opJ_\opB $:
\begin{align*}
(2\opJ_{\opA ^{-\ovee}}-\opI )(2\opJ_{\opB ^{-1}}-\opI )&=(2\opJ_\opA (-\opI )+\opI )(\opI -2\opJ_\opB )\\
&=(2\opJ_\opA (-\opI )+\opI ) (-\opI )(2\opJ_\opB -\opI )\\
&=(2\opJ_\opA -\opI )(2\opJ_\opB -\opI )
\end{align*}
\end{frame}

\begin{frame}
\frametitle{Self-dual property of DRS}
When $\alpha=1$, DRS is:
\begin{align*}
x^{k+1/2}&=\opJ_{\opB }(z^k)\\
 u ^{k+1/2}&=\opJ_{\opB ^{-1}}(z^k)=z^k-x^{k+1/2}\\
x^{k+1}&=\opJ_{ \opA }(2x^{k+1/2}-z^k)\\
 u ^{k+1}&=\opJ_{\opA ^{-\ovee}}(2 u ^{k+1/2}-z^k)=x^{k+1}-x^{k+1/2}+ u ^{k+1/2}\\
z^{k+1}&=z^k+x^{k+1}-x^{k+1/2}=z^k+ u ^{k+1}- u ^{k+1/2}
\end{align*}
Nicely reveals the symmetry.
(Algorithmically no need to use both the $x$ and $ u $.)
When $\alpha\ne 1$, similar but less elegant self-dual form.

\vspace{0.2in}



%When solving an optimization problem with ADMM or DRS, one can often apply ADMM or DRS to the primal or the dual problems.
%Although the two options may result in seemingly different algorithms, they are usually equivalent due to the self-dual property of DRS.
(This self-dual property explains why the infimal postcomposition technique and the dualization technique yield the same ADMM.)
\end{frame}

\begin{frame}
\frametitle{DYS}
For 
\[ 
\begin{array}{ll}
\underset{x\in\reals^n}{\mathrm{find}} &
0\in (\opA+\opB+\opC)x,
\end{array}
\]
where $\opA$, $\opB$, and $\opC$ are maximal monotone and $\opC$ is single-valued,
 Attouch--Th\'era dual is
\[ 
\begin{array}{ll}
\underset{ u \in\reals^n}{\mathrm{find}} &
0\in ((\opA+\opC)^{-\ovee}+\opB^{-1}) u .
\end{array}
\]
\vspace{0.2in}

Fixed points of DYS:
\begin{align*}
\fix&\left(\opI-\opJ_{\alpha \opB}+\opJ_{\alpha \opA}(\opR_{\alpha \opB}-\alpha \opC\opJ_{\alpha \opB})\right)\\
&\qquad\qquad\qquad\qquad\subseteq\zer(\opA+\opB+\opC)+\alpha \zer((\opA+\opC)^{-\ovee}+\opB^{-1}).
\end{align*}
\end{frame}

\begin{frame}[plain]
\frametitle{Primal-dual DYS}

We can make the FPI with DYS more explicitly primal-dual:
\begin{align*}
x^{k+1/2}&=\opJ_{\alpha \opB}(z^k)\\
 u ^{k+1/2}&=\frac{1}{\alpha}(z^k-x^{k+1/2})\\
x^{k+1}&=\opJ_{\alpha A}(2x^{k+1/2}-z^k-\alpha \opC x^{k+1/2})\\
 u ^{k+1}&=\frac{1}{\alpha}(x^{k+1}-x^{k+1/2}+\alpha u ^{k+1/2})\\
z^{k+1}&=z^k+x^{k+1}-x^{k+1/2}.
\end{align*}
Note $ u ^{k+1/2}\in \opB x^{k+1/2}$, $- u ^{k+1}\in \opA x^{k+1}+\opC x^{k+1/2}$.
If $z^k\rightarrow z^\star$, then
\begin{gather*}
x^{k+1/2}\rightarrow x^\star,\quad x^{k+1}\rightarrow x^\star,\quad
x^\star\in \zer(\opA+\opB+\opC)\\
 u ^{k+1/2}\rightarrow  u ^\star
,\quad
 u ^{k+1}\rightarrow  u ^\star,\quad
 u ^\star\in \zer((\opA+\opC)^{-\ovee}+\opB^{-1})\\
z^k\rightarrow x^\star+\alpha u ^\star.
\end{gather*}
\vspace{0.0in}

DYS is not self-dual as it uses an evaluation of $\opC$, a primal operation.
%There seems to be no elegant way of describing the DYS operator using the dual operators $\opB^{-1}$ and $(\opA+\opC)^{-\ovee}$ rather than the primal operators $\opA$, $\opB$, and $\opC$.
\end{frame}


\end{document}
