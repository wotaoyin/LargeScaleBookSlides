\documentclass[10pt,mathserif]{beamer}

\usepackage{graphicx,amsmath,amssymb,psfrag,mathtools}
\usepackage{soul}
\usepackage{amsmath,amsfonts,amsthm,bbm}
\usepackage{stmaryrd}
\usepackage{subcaption}



%Code block environment
\usepackage{listings}


\definecolor{lightgrey}{gray}{0.8}
\definecolor{medgrey}{gray}{0.6}
\definecolor{darkgrey}{gray}{0.4}
\usepackage{xcolor}
\lstset { %
    backgroundcolor=\color{black!5}, % set backgroundcolor
    basicstyle=\ttfamily,
    showstringspaces=false,
    commentstyle = \ttfamily,
    commentstyle=\color{commentgreen}\ttfamily,
    morecomment=[l][\color{darkgrey}]{//},
}



\usepackage{tikz}
\usetikzlibrary{matrix,chains,positioning,decorations.pathreplacing,arrows}
\usetikzlibrary{positioning,calc}
\usepackage{tkz-euclide}
%\usetkzobj{all}





%-------------------------------------------------------------------------------
%Definition of operator font
 \usepackage[bb=boondox]{mathalfa}
%% import \varmathbb without affecting other fonts
\usepackage{xparse}
\DeclareFontFamily{U}{ntxmia}{}
\DeclareFontShape{U}{ntxmia}{m}{it}{<-> ntxmia }{}
\DeclareFontShape{U}{ntxmia}{b}{it}{<-> ntxbmia }{}
\DeclareSymbolFont{lettersA}{U}{ntxmia}{m}{it}
\SetSymbolFont{lettersA}{bold}{U}{ntxmia}{b}{it}
\ExplSyntaxOn
\NewDocumentCommand{\varmathbb}{m}
 {
  \tl_map_inline:nn { #1 }
   {
    \use:c { varbb##1 }
   }
 }
\tl_map_inline:nn { ABCDEFGHIJKLMNOPQRSTUVWXYZ }
 {
  \exp_args:Nc \DeclareMathSymbol{varbb#1}{\mathord}{lettersA}{\int_eval:n { `#1+67 }}
 }
\exp_args:Nc \DeclareMathSymbol{varbbk}{\mathord}{lettersA}{169}
\ExplSyntaxOff
%%
\makeatletter
\DeclareFontFamily{U}{tipa}{}
\DeclareFontShape{U}{tipa}{m}{n}{<->tipa10}{}
\newcommand{\arc@char}{{\usefont{U}{tipa}{m}{n}\symbol{62}}}%

\newcommand{\arc}[1]{\mathpalette\arc@arc{#1}}

\newcommand{\arc@arc}[2]{%
  \sbox0{$\m@th#1#2$}%
  \vbox{
    \hbox{\resizebox{\wd0}{\height}{\arc@char}}
    \nointerlineskip
    \box0
  }%
}
\makeatother
\newcommand{\opA}{{\varmathbb{A}}}
\newcommand{\opB}{{\varmathbb{B}}}
\newcommand{\opC}{{\varmathbb{C}}}
\newcommand{\opD}{{\varmathbb{D}}}
\newcommand{\opE}{{\varmathbb{E}}}
\newcommand{\opF}{{\varmathbb{F}}}
\newcommand{\opG}{{\varmathbb{G}}}
\newcommand{\opH}{{\varmathbb{H}}}
\newcommand{\opI}{{\varmathbb{I}}}
\newcommand{\opJ}{{\varmathbb{J}}}
\newcommand{\opK}{{\varmathbb{K}}}
\newcommand{\opL}{{\varmathbb{L}}}
\newcommand{\opM}{{\varmathbb{M}}}
\newcommand{\opN}{{\varmathbb{N}}}
\newcommand{\opO}{{\varmathbb{O}}}
\newcommand{\opP}{{\varmathbb{P}}}
\newcommand{\opQ}{{\varmathbb{Q}}}
\newcommand{\opR}{{\varmathbb{R}}}
\newcommand{\opS}{{\varmathbb{S}}}
\newcommand{\opT}{{\varmathbb{T}}}
\newcommand{\opU}{{\varmathbb{U}}}
\newcommand{\opV}{{\varmathbb{V}}}
\newcommand{\opW}{{\varmathbb{W}}}
\newcommand{\opX}{{\varmathbb{X}}}
\newcommand{\opY}{{\varmathbb{Y}}}
\newcommand{\opZ}{{\varmathbb{Z}}}
\newcommand{\opZer}{\mathbb{0}}
%-------------------------------------------------------------------------------



%-------------------------------------------------------------------------------
%Definition of other font types
\newcommand{\va}{{\mathbf{a}}}
\newcommand{\vb}{{\mathbf{b}}}
\newcommand{\vc}{{\mathbf{c}}}
\newcommand{\vd}{{\mathbf{d}}}
\newcommand{\ve}{{\mathbf{e}}}
\newcommand{\vf}{{\mathbf{f}}}
\newcommand{\vg}{{\mathbf{g}}}
\newcommand{\vh}{{\mathbf{h}}}
\newcommand{\vi}{{\mathbf{i}}}
\newcommand{\vj}{{\mathbf{j}}}
\newcommand{\vk}{{\mathbf{k}}}
\newcommand{\vl}{{\mathbf{l}}}
\newcommand{\vm}{{\mathbf{m}}}
\newcommand{\vn}{{\mathbf{n}}}
\newcommand{\vo}{{\mathbf{o}}}
\newcommand{\vp}{{\mathbf{p}}}
\newcommand{\vq}{{\mathbf{q}}}
\newcommand{\vr}{{\mathbf{r}}}
\newcommand{\vs}{{\mathbf{s}}}
\newcommand{\vt}{{\mathbf{t}}}
\newcommand{\vu}{{\mathbf{u}}}
\newcommand{\vv}{{\mathbf{v}}}
\newcommand{\vw}{{\mathbf{w}}}
\newcommand{\vx}{{\mathbf{x}}}
\newcommand{\vy}{{\mathbf{y}}}
\newcommand{\vz}{{\mathbf{z}}}

\newcommand{\vA}{{\mathbf{A}}}
\newcommand{\vB}{{\mathbf{B}}}
\newcommand{\vC}{{\mathbf{C}}}
\newcommand{\vD}{{\mathbf{D}}}
\newcommand{\vE}{{\mathbf{E}}}
\newcommand{\vF}{{\mathbf{F}}}
\newcommand{\vG}{{\mathbf{G}}}
\newcommand{\vH}{{\mathbf{H}}}
\newcommand{\vI}{{\mathbf{I}}}
\newcommand{\vJ}{{\mathbf{J}}}
\newcommand{\vK}{{\mathbf{K}}}
\newcommand{\vL}{{\mathbf{L}}}
\newcommand{\vM}{{\mathbf{M}}}
\newcommand{\vN}{{\mathbf{N}}}
\newcommand{\vO}{{\mathbf{O}}}
\newcommand{\vP}{{\mathbf{P}}}
\newcommand{\vQ}{{\mathbf{Q}}}
\newcommand{\vR}{{\mathbf{R}}}
\newcommand{\vS}{{\mathbf{S}}}
\newcommand{\vT}{{\mathbf{T}}}
\newcommand{\vU}{{\mathbf{U}}}
\newcommand{\vV}{{\mathbf{V}}}
\newcommand{\vW}{{\mathbf{W}}}
\newcommand{\vX}{{\mathbf{X}}}
\newcommand{\vY}{{\mathbf{Y}}}
\newcommand{\vZ}{{\mathbf{Z}}}

\newcommand{\cA}{{\mathcal{A}}}
\newcommand{\cB}{{\mathcal{B}}}
\newcommand{\cC}{{\mathcal{C}}}
\newcommand{\cD}{{\mathcal{D}}}
\newcommand{\cE}{{\mathcal{E}}}
\newcommand{\cF}{{\mathcal{F}}}
\newcommand{\cG}{{\mathcal{G}}}
\newcommand{\cH}{{\mathcal{H}}}
\newcommand{\cI}{{\mathcal{I}}}
\newcommand{\cJ}{{\mathcal{J}}}
\newcommand{\cK}{{\mathcal{K}}}
\newcommand{\cL}{{\mathcal{L}}}
\newcommand{\cM}{{\mathcal{M}}}
\newcommand{\cN}{{\mathcal{N}}}
\newcommand{\cO}{{\mathcal{O}}}
\newcommand{\cP}{{\mathcal{P}}}
\newcommand{\cQ}{{\mathcal{Q}}}
\newcommand{\cR}{{\mathcal{R}}}
\newcommand{\cS}{{\mathcal{S}}}
\newcommand{\cT}{{\mathcal{T}}}
\newcommand{\cU}{{\mathcal{U}}}
\newcommand{\cV}{{\mathcal{V}}}
\newcommand{\cW}{{\mathcal{W}}}
\newcommand{\cX}{{\mathcal{X}}}
\newcommand{\cY}{{\mathcal{Y}}}
\newcommand{\cZ}{{\mathcal{Z}}}
%-------------------------------------------------------------------------------




%-------------------------------------------------------------------------------
%% macros for math notions and operators
\newcommand{\EE}{{\mathbb{E}}}
\newcommand{\expec}{\mathbb{E}}
\newcommand{\Prob}{{\mathrm{Prob}}} % probability

\newcommand{\reals}{\mathbb{R}}
\newcommand{\RR}{\mathbb{R}}
\newcommand{\complex}{\mathbb{C}}
\newcommand{\CC}{\mathbb{C}}
\newcommand{\nats}{\mathbb{N}}
\newcommand{\NN}{\mathbb{N}}
\newcommand{\ZZ}{\mathbb{Z}}
\newcommand{\bigO}{\mathcal{O}}
\newcommand{\order}[1]{{\mathcal{O}\left(#1\right)}}
\renewcommand{\SS}{{\mathbb{S}}}
\newcommand{\SSp}{\mathbb{S}_{+}}
\newcommand{\SSpp}{\mathbb{S}_{++}}
\newcommand{\sign}{\mathrm{sign}}
\newcommand{\vzero}{\mathbf{0}}
\newcommand{\vone}{{\mathbf{1}}}

\renewcommand{\Re}{\operatorname{Re}} 	%Real part
\renewcommand{\Im}{\operatorname{Im}}	%imaginary part

%\newcommand{\supp}{{\mathrm{supp}}} % support
\newcommand{\range}{\mathrm{range}\,} % domain
\newcommand{\tr}{{\mathrm{tr}}} % trace
%-------------------------------------------------------------------------------





%-------------------------------------------------------------------------------
%% Theorem definitions
\setbeamertemplate{theorems}[ams style] 
\newtheorem*{theorem*}{Theorem}
%\newtheorem{lemma}{Lemma}    % already provided by amsthm
\newtheorem{proposition}{Proposition}
%\newtheorem{proof}{Proof}  % already provided by amsthm


%-------------------------------------------------------------------------------
%% operator and convex analysis definitions

\newcommand*{\fix}{\mathrm{Fix}\,}
\newcommand*{\zer}{\mathrm{Zer}\,}
\newcommand*{\gra}{\mathrm{Gra}\,}
\newcommand{\prox}{\mathrm{Prox}}
\newcommand{\proj}{\Pi}
\newcommand{\aff}{\mathrm{aff}\,}    %affine hull
\newcommand{\intr}{\mathrm{int}\,}   %interior
\newcommand{\relint}{\mathrm{ri}\,}  %relative interior
\newcommand{\dom}{\mathrm{dom}\,} % domain
\newcommand{\epi}{\mathrm{epi}\,} % epigraph
\newcommand{\dist}{\mathrm{dist}}
\newcommand{\lagrange}{\mathbf{L}}  %saddle function
\newcommand{\fitzpatrick}{\mathbf{F}}   %Fitzpatrick function
\newcommand{\vecdelay}{\boldsymbol{d}}   %vector delay
\DeclareMathOperator*{\argmin}{argmin}
\DeclareMathOperator*{\argmax}{argmax}

%-------------------------------------------------------------------------------
%SRG definitions
\newcommand{\ereal}{\overline{\mathbb{R}^2}}
\newcommand{\ecomplex}{\overline{\mathbb{C}}}
\newcommand{\binfty}{{\boldsymbol \infty}}
\newcommand{\rarc}{\mathrm{Arc}^+}
\newcommand{\larc}{\mathrm{Arc}^-}


%-------------------------------------------------------------------------------




%-------------------------------------------------------------------------------
%Miscellaneous Stuff
%% sequences
\newcommand{\itom}{_{i=1}^{m}}
\newcommand{\ieqm}{i=1,\dots,m}

% use \numberthis to add an equation number in align*
\newcommand\numberthis{\addtocounter{equation}{1}\tag{\theequation}}

\newcolumntype{P}[1]{>{\centering\arraybackslash}p{#1}}

\mode<presentation>
{
\usetheme{default}
}
\setbeamertemplate{navigation symbols}{}
\usecolortheme[rgb={0.13,0.28,0.59}]{structure}
\setbeamertemplate{itemize subitem}{--}
\setbeamertemplate{frametitle} {
	\begin{center}
	  {\large\bf \insertframetitle}
	\end{center}
}

\newcommand\footlineon{
  \setbeamertemplate{footline} {
    \begin{beamercolorbox}[ht=2.5ex,dp=1.125ex,leftskip=.8cm,rightskip=.6cm]{structure}
      \footnotesize \insertsection
      \hfill
      {\insertframenumber}
    \end{beamercolorbox}
    \vskip 0.45cm
  }
}
\footlineon


\newcommand\footlineoff{
  \setbeamertemplate{footline} {
    \begin{beamercolorbox}[ht=2.5ex,dp=1.125ex,leftskip=.8cm,rightskip=.6cm]{structure}
      \footnotesize 
      \hfill
      {\insertframenumber}
    \end{beamercolorbox}
    \vskip 0.45cm
  }
}


\newcommand\blfootnote[1]{%
  \begingroup
  \renewcommand\thefootnote{}\footnote{#1}%
  \addtocounter{footnote}{-1}%
  \endgroup
}


\AtBeginSection[] 
{ 
	\begin{frame}<beamer> 
		\frametitle{Outline} 
		\tableofcontents[currentsection,currentsubsection] 
	\end{frame} 
} 

%% wotao's preference

        % itemize, black bullet, %150 spacing between items using "witemize"
        \newenvironment{witemize}{\itemize\addtolength{\itemsep}{0.3\baselineskip}}{\enditemize}

\iffalse
        % \setbeamertemplate{itemize items}[square]
        \setbeamertemplate{itemize items}{\textbullet}
        \setbeamercolor{itemize item}{fg=black}
        \setbeamercolor{itemize subitem}{fg=black}
        \setbeamercolor{itemize subsubitem}{fg=black}
        \setbeamercolor{enumerate item}{fg=black}
        \setbeamercolor{enumerate subitem}{fg=black}
        \setbeamercolor{enumerate subsubitem}{fg=black}
        \setbeamertemplate{itemize/enumerate body begin}{\normalsize}
        \setbeamertemplate{itemize/enumerate subbody begin}{\normalsize}
        \setbeamertemplate{itemize/enumerate subsubbody begin}{\normalsize}

        % itemize enumerate use normal sized texts
        \setbeamertemplate{itemize/enumerate body begin}{\normalsize}
        \setbeamertemplate{itemize/enumerate subbody begin}{\normalsize}
        \setbeamertemplate{itemize/enumerate subsubbody begin}{\normalsize}

        % block, black over gray with no shadow
        \setbeamertemplate{blocks}[rounded][shadow=false]
        \setbeamercolor{block title}{fg=black,bg=gray!40}
        \setbeamercolor{block body}{fg=black,bg=gray!10}
\fi

\author{Ernest K. Ryu and Wotao Yin}

\date{Large-Scale Convex Optimization via Monotone Operators}


\title{\large \bfseries Randomized Coordinate Update Methods}

\begin{document}

\frame{
\thispagestyle{empty}
\titlepage
}



\section{Randomized coordinate fixed-point iteration}


\begin{frame}[fragile,plain]
\frametitle{Coordinate partitioning}
Partition $x\in \reals^n$ into $m$ non-overlapping blocks of sizes $n_1,\dots,n_m$.
\medskip\pause 

Write\footnote{some researchers would write $x=[x_1;\dots; x_m]$ instead} $x=(x_1,\dots,x_m)$,
so $x_i\in \reals^{n_i}$.
\medskip

Partition $\opT\colon\reals^n\rightarrow\reals^n$ into $(\opT(x))_i\in \reals^{n_i}$ and define $\opT_i(x)$ as follows:
\[
\opT(x)=\begin{bmatrix}
(\opT(x))_1\\
\vdots\\
(\opT(x))_i\\
\vdots\\
(\opT(x))_m
\end{bmatrix}
\qquad
% \]
% \pause Define
% \begingroup\makeatletter\def\f@size{8}\check@mathfonts
% \[
\begingroup
\renewcommand*{\arraystretch}{.8}
\opT_i(x)=\begin{bmatrix}
x_1\\
\vdots\\
x_{i-1}\\
(\opT(x))_i\\
x_{i+1}\\
\vdots\\
x_m
\end{bmatrix},
\endgroup
\]
% \endgroup
i.e., $\opT_i$ is $\opT$ on the $i$\nobreakdash-th block and is identity on the other blocks.

\medskip
We say ``block''  and ``coordinate'' interchangeably.
\end{frame}




\begin{frame}
\frametitle{Coordinate-update fixed-point iteration}
For $\opT\colon\reals^n\rightarrow\reals^n$, consider 
\begin{align*}
\begin{array}{ll}
\underset{x\in\reals^n}{\mbox{find}}&
x = \opT x.
\end{array}
\end{align*}
\vspace{0.2in}

Coordinate-update fixed-point iteration (C-FPI) is
\begin{align*}
\text{select}&\,\,\,i(k) \in \{1,\dots,m\},\\
%i(k)&\sim \mathrm{Uniform}(\{1,2,\dots,n\})\\
% &x^{k+1}_i = \begin{cases}
%               T_{i}(x^k), & \mbox{if } i=i(k) \\
%               T_i, & \mbox{otherwise},
%             \end{cases}
%             \qquad \text{for}~i \in[m].
x^{k+1}&= \opT_{i(k)}(x^k).
\end{align*}
At the $k$\nobreakdash-th iteration, C-FPI updates only the $i(k)$\nobreakdash-th block.
\medskip

Specifying the selection rule for $i(k)$ fully specifies the method.
\end{frame}




\begin{frame}
\frametitle{Block selection rules}
There are many ways to select $i(k)$ with different advantages and disadvantages.
%and different rules have different convergence properties and can take advantage of different problem structures and computational setups.

\vspace{0.2in}

Common selection rules:
\begin{itemize}
\item\pause
Cyclic rule. Select the blocks in a fixed cyclic order.
\item\pause
Cyclic shuffle rule. Shuffle the ordering of blocks and select each of them sequentially, and repeat. A.k.a. random sample without replacement.
\item\pause
Essential cyclic rule. Each block appears once or more in each ``cycle''. (Both cyclic and cyclic shuffle rules are special cases.)
\item\pause
Greedy rule. Select block that leads to the most immediate progress, measured in many different ways.

\item\pause
Randomized rule. Select blocks randomly. Each selection is independent of others. A.k.a. random sample with replacement.
\end{itemize}

%The \emph{essentially cyclic rule} selects every coordinate at least once every fixed number of iterations. The \emph{randomized rule} selects a coordinate randomly, where different coordinates may be selected for a different probability. The \emph{greedy rule} selects a coordinate that leads to the greatest progress, where the progress can be measured in many different ways. %There combinations of conditions and rules that can be proved to work, and the list still grows.
\end{frame}

\begin{frame}
\frametitle{Randomized coordinate-update fixed-point iteration}

We focus on the randomized rule (i.e. select $i(k)\in\{1,\dots,m\}$ independently uniformly at random) since its analysis is simplest.

\vspace{0.2in}

We get randomized coordinate-update fixed-point iteration (RC-FPI):
\begin{align*}
i(k)&\sim \text{IID Uniform}\{1,\dots,m\}\\
x^{k+1}&= \opT_{i(k)}(x^k).
\end{align*}

\end{frame}

\begin{frame}
\frametitle{Randomized coordinate-update fixed-point iteration}

\setcounter{theorem}{1}
\begin{theorem}
Assume $\opT \colon\reals^n\rightarrow\reals^n$ is $\theta$-averaged with $\theta\in(0,1)$ and $\fix \opT \ne \emptyset$.
Assume the random indices $i(0),i(1),\ldots\in \{1,\dots,m\}$ are i.i.d. with uniform probability.
Then $x^{k+1}=\opT _{i(k)}x^k$
with any starting point $x^0\in \reals^n$
converges to one fixed point with probability 1, i.e., the event
\[
x^k\rightarrow x^\star
\]
holds 
with probability 1 for some $x^\star\in \fix \opT$.
The quantities
$ \expec\, \dist^2(x^{k},\fix \opT )$ and 
% $\expec\|\opT x^{k}-x^k\|^2$, and
$\expec\|x^{k}-x^\star\|^2$  for any $x^\star\in \fix \opT $ decrease monotonically with $k$.
Finally, we have
\[
\dist(x^{k},\fix \opT )\rightarrow 0
\]
with probability 1.
% and
% \begin{align*}
% \expec\|x^{k+1}-x^k\|^2\le \frac{\theta m}{(k+1)(1-\theta)}\dist^2(x^{0},\fix \opT ).
% \end{align*}
% \vspace{-0.2in}
\end{theorem}
\end{frame}
\begin{frame}
\frametitle{Prepare to prove Theorem 2}
Proving Theorem 2 requires the following standard result from probability theory.
\begin{theorem*}
[supermartingale convergence theorem]
% Let $V^k$ and $S^k$ be $\mathcal{F}_k$-measurable random variables satisfying $V^k\ge 0$ and $S^k\ge 0$ almost surely for $k=0,1,\dots$.
Let $(V^k)_{k=0,1,\dots}$ and $(S^k)_{k=0,1,\dots}$ be two sequences of random variables satisfying $V^k\ge 0$ and $S^k\ge 0$ almost surely for $k=0,1,\dots$.
Assume 
\[
\expec\left[V^{k+1}\,|\,V^0,\dots,V^k\right]\le V^k - S^k
\]
holds for $k=0,1,\dots$.
Then
\begin{enumerate}
    \item $\sum^\infty_{k=0}S^k<\infty$
    \item $V^k\rightarrow V^\infty$
\end{enumerate}
hold with probability 1.
(Note that the limit $V^\infty$ is a random variable.)
\end{theorem*}
\end{frame}

\begin{frame}[fragile]
\frametitle{Prepare to prove Theorem 2}
Define $\opS $ with $\opT  = \opI  - \theta \opS $ and $\opS _i$ with $\opT _i=\opI -\theta \opS _i$.
So we have
\begingroup\makeatletter\def\f@size{8}\check@mathfonts
\[
\begingroup
\renewcommand*{\arraystretch}{.8}
\opS _i(x)=
\begin{bmatrix}
0\\
\vdots\\
0\\
(\opS (x))_i\\
0\\
\vdots\\
0
\end{bmatrix}
\endgroup
\]
\endgroup
for $i=1,\dots,m$, and $\sum_{i=1}^m \opS_i = \opS$.

% Define $\opS _i$ such that
% \begin{align*}
%   \opT _ix^k = x^k - \theta \opS _i(x^k)
% \end{align*}
% for all $i\in [m]$.

\medskip\pause

Alternately express $x^{k+1}=\opT _{i(k)}x^k$ as
\begin{align*}
  x^{k+1}&= x^k - \theta \opS _{i(k)}x^k.
\end{align*}
\end{frame}

\begin{frame}[fragile]
\frametitle{Prepare to prove Theorem 2}
Fact: $\opT $ is $\theta$-averaged if and only if $\opS $ is $(1/2)$-cocoercive. Proof:
\begin{align*}
\opT \text{ is $\theta$-averaged}
\quad&\Leftrightarrow\quad
\frac{1}{\theta}\opT -\left(\frac{1}{\theta}-1\right)\opI \text{ is nonexpansive}\\
&\Leftrightarrow\quad
\opI -\opS  \text{ is nonexpansive}\\
&\Leftrightarrow\quad
\|x-\opS x-y+\opS y\|^2\le \|x-y\|^2\quad\forall\, x,y\in \reals^n\\
&\Leftrightarrow\quad
\frac{1}{2}\|\opS x-\opS y\|^2\le \langle x-y,\opS x-\opS y\rangle\quad\forall\, x,y\in \reals^n\\
&\Leftrightarrow\quad
\opS \text{ is $(1/2)$-cocoercive}.
\end{align*}
\pause Clearly, $\fix\opT=\zer\opS$.
For any $x^\star\in \fix \opT =\zer\opS$ and $x\in \reals^n$,
\begin{align}\label{eq:Rcocoer}
\frac{1}{2}\|\opS  x\|^2\le   \langle \opS  x, x-x^\star\rangle
\end{align}
\end{frame}

\begin{frame}
\frametitle{Prepare to prove Theorem 2}
$x^{k+1}$ is a random variable depending on $i(k),i(k-1),\dots,i(0)$.
\\
 $x^0$ is arbitrary but not random.
Write $\EE$ for the full expectation.
Write $\EE_k$ for the conditional expectation with respect to $i(k)$ conditioned on the past random variables $i(k-1),i(k-2),\dots,i(0)$.
%The law of total expectation states that $\EE\big[\EE_k[X]\big] = \EE[X]$, if $X$ is a random variable such that $\EE[X]$ is well-defined.
%Then $\EE\big[\EE_k[X]\big] = \EE[X]$ by the law of total expectation.

\vspace{0.2in}\pause
Under these definitions,  $\EE_k[x^k] = x^k$ and
\begin{align}
  \EE_{k} [\opS_{i(k)}x^k] & = \sum_{i=1}^m \frac{1}{m}\opS_ix^k= \frac{1}{m} \opS x^k, \label{eq:RiR} \\
  \EE_{k} \|\opS_{i(k)}x^k\|^2 & = \sum_{i=1}^m \frac{1}{m}\|\opS_ix^k\|^2 = \frac{1}{m} \|\opS x^k\|^2. \label{eq:RiR2}
\end{align}
\end{frame}

\begin{frame}
\frametitle{Proof of Theorem 2}
\textbf{Stage 1.}
For any $x^\star\in\fix \opT$, 
\begin{align*}
  \|x^{k+1} - x^\star\|^2 & = \|x^k - \theta \opS_{i(k)}x^k - x^\star\|^2 \nonumber\\
  & = \|x^k - x^\star\|^2 - 2\theta\langle \opS_{i(k)}x^k ,x^k - x^\star\rangle + \theta^2 \|\opS_{i(k)}x^k\|^2.
\end{align*}
%using the formula of $x^{k+1}$, we get
Take conditional expectation $\EE_k$ and use \eqref{eq:RiR} and \eqref{eq:RiR2}:
\begin{align*}
  \EE_k \|x^{k+1} - x^\star\|^2 &  = \|x^k - x^\star\|^2 - 2\theta\langle \EE_k [\opS_{i(k)}x^k] ,x^k - x^\star\rangle + \theta^2 \EE_k\|\opS_{i(k)}x^k\|^2\nonumber\\
  & = \|x^k - x^\star\|^2 - \frac{2\theta}{m}\langle \opS x^k,x^k - x^\star\rangle + \frac{\theta^2}{m}\|\opS x^k\|^2\\
  & \le\|x^k - x^\star\|^2 - (1-\theta)\frac{\theta}{m}\|\opS x^k\|^2, \numberthis \label{eq:condmon}
\end{align*}
where the inequality follows from \eqref{eq:Rcocoer}.
\vspace{0.2in}

So $(\|x^{k} - x^\star\|^2)_{k=0,1,\ldots}$ a nonnegative supermartingale.

\end{frame}

\begin{frame}
\frametitle{Proof of Theorem 2}
Take the full expectation on both ends of \eqref{eq:condmon}:
\begin{align*}
  \EE\|x^{k+1} - x^\star\|^2 & \le \EE\|x^{k} - x^\star\|^2 - (1-\theta)\frac{\theta}{m}\EE\|\opS x^{k}\|^2.
\end{align*}
\vspace{0.2in}

Therefore, $\EE\|x^{k} - x^\star\|^2$ decreases monotonically with $k$ and, by minimizing over $x^\star \in \fix \opT$, so does $\EE \,\dist^2(x^{k},\fix \opT)$.
\end{frame}

\begin{frame}
\frametitle{Proof of Theorem 2}
\textbf{Stage 2.}
We prove convergence of the iterates.
Apply the \emph{supermartingale convergence theorem}
%of \cite[Theorem 1]{RobbinsSiegmund1985_ConvergenceTheorem},
to \eqref{eq:condmon} to get:
\begin{enumerate}
  \item[(i)] $\sum_{k=0}^\infty\|\opS x^k\|^2 < \infty$ and 
  \item[(ii)] $\lim_{k\rightarrow\infty}\|x^k - x^\star\|$ exists
%  \item $x^{k}$ is bounded,
\end{enumerate}
 with probability 1.
Note (i) implies $\|\opS x^k\|^2\rightarrow 0$ and (ii) implies $x^{k}$ is bounded with probability 1.

\vspace{0.2in}\pause
For all $x^\star\in \fix\opT$, $\lim_{k\rightarrow\infty}\|x^k - x^\star\|$ exists with probability 1.
Apply Proposition~\ref{prp:prob1} (stated and proved later) to conclude  with probability 1, $\lim_{k\rightarrow\infty}\|x^k - x^\star\|$ exists for all $x^\star\in \fix \opT$.
%The distinction between statements 1 and 2 of Proposition~\ref{prp:prob1} becomes necessary when $\fix T$ is not a singleton and therefoer has uncountably many fixed points.
Apply the same argument of Theorem~1 to get: with probability 1, $x^k\rightarrow x^\star$.
% is a random variable that depends on $x^\star$ and the random indices $i(0),i(1),\dots$.
\qed
\end{frame}

\begin{frame}[plain]
\frametitle{Measurability argument}
%Proposition~\ref{prp:prob1} is subtle.
In the proof, we first choose an arbitrary $x^\star\in \fix \opT$ and then apply the supermartingale convergence theorem, so [$\lim_{k\rightarrow\infty}\|x^k - x^\star\|$ holds with probability 1] applies to each $x^\star$.
\medskip

We need a stronger statement before we can the argument of Theorem~1: [$\lim_{k\rightarrow\infty}\|x^k - x^\star\|$ for all $x^\star\in \fix \opT$] holds with probability 1.
\medskip\pause

The subtlety: 
\begin{enumerate}
    \item the intersection of \emph{countably many} measure-1 sets has measure 1;
    \item the intersection of \emph{uncountably many} measure-1 sets has measure \emph{less than or} equal to 1.
\end{enumerate}
As set $\fix \opT$ is possibly uncountable, case 2 is applicable.
\medskip\pause

Proposition 1 (next slide) shows the stronger statement actually holds.
\end{frame}

\begin{frame}{Proposition 1}
%\setcounter{proposition}{1}
\begin{proposition}\label{prp:prob1}
Let $Y\subseteq \reals^n$ and let $x^0,x^1,\dots$ be a random sequence. Then statement 1 implies statement 2.
\begin{enumerate}
  \item For all $y\in Y$ \emph{[with probability 1, $\lim_{k\rightarrow\infty}\|x^k - y\|$ exists]}.
  \item With probability 1 \emph{[for all $y\in Y$, $\lim_{k\rightarrow\infty}\|x^k - y\|$ exists]}.
\end{enumerate}
\end{proposition}
\textbf{Proof outline.}
(i) $Y\subseteq \reals^n$ is separable (i.e., has a countable dense subset), (ii) sequence of functions $\{\|x^k-\cdot \|\}_{k\in \mathbb{N}}$ has a limit on the countable dense subset of $Y$, and (iii) if an equicontinuous sequence of functions has limits on the dense subset of $Y$, limits exist on all of $Y$.\qed 
\medskip

Understand ``equicontinuous'' as ``uniformly (over sequence) uniformly (over function domain) continuous''
\end{frame}




\section{Coordinate and extended coordinate friendly operators}

\begin{frame}
\frametitle{Coordinate friendly operators}

While Theorem~2 is true, RC-FPI is useful only if it is computationally efficient.
\medskip

RC-FPI is computationally effective when $\opT$ is coordinate friendly (defined below) or extended coordinate friendly (defined later).

\vspace{0.2in}\pause

Let $z=(z_1,\dots,z_m)\in\reals^n$.
% and $z_i\in\reals^{n_i}$.
Then $x\mapsto z$ is coordinate friendly if
\[
%\frac{m}{p}
\max_{i=1,\dots,m}
\mathcal{F}[x\mapsto z_i]
\ll
\mathcal{F}[x\mapsto z].
\]
%More precisely, a method is coordinate friendly if for all $i=1,\dots,m$ there is an algorithm for $x\mapsto y_i$ that is much faster than the serial algorithms for $x\mapsto y$.
(Meaning of $\ll$ depends on context.)
%\[
%\mathcal{F}[x\mapsto y_i]
%\sim \frac{C}{m}\mathcal{F}[x\mapsto \{y_1,\dots,y_m\}]
%\quad
%\text{ for }i=1,\dots,m
%\]
%for some $C>0$ not too large, we safely say the method is coordinate friendly.
%Again, some authors use the terminology `block coordinate friendly' and reserve `coordinate friendly' for when $n_1=\dots=n_m=1$, but we do not make this distinction.

\vspace{0.2in}
$\opT$ is coordinate friendly if  $x\mapsto \opT x$ is coordinate friendly.


\end{frame}

\begin{frame}
\frametitle{Coordinate friendly $\Rightarrow$ parallelizable}
If $x\mapsto z$ is coordinate friendly, then for $p\ge m$
\[
\mathcal{F}_p[x\mapsto 
z]
%\{z_1,\dots,z_m\}]
=
%\frac{m}{p}
\max_{i=1,\dots,m}
\mathcal{F}[x\mapsto z_i]
\ll
\mathcal{F}[x\mapsto 
z].
%\{z_1,\dots,z_m\}]
\]
So $x\mapsto z$ is parallelizable.



\end{frame}



\begin{frame}
\frametitle{Affine operators}
Affine operator $\opT x = Ax+b$, where $A\in \reals^{n\times n}$ and $b\in \reals^n$, is coordinate friendly if $n_i\ll n$ for $i=1,\dots,m$, since
\begin{gather*}
 \mathcal{F}[x\mapsto \opT_i x]\sim 2nn_i
 \quad\ll\quad
\mathcal{F}[x\mapsto \opT x]\sim 2n^2.
\end{gather*}
\end{frame}
% is CF since $\mathcal{F}[x\mapsto T_i x]=m$ and $\mathcal{F}[x\mapsto Tx]=m^2$.


% For example, $x^{k+1}=x^k-Mx^k$, where $M\in \reals^{n\times n}$ is coordinate friendly, since

\begin{frame}
\frametitle{Separable operators}
%Let $n_1+\dots+n_m=n$ and write $\vx=(x_1,\dots,x_m)$ where $x_i\in \reals^{n_i}$ for $\ieqm$.
$\opT\colon\reals^n\rightarrow \reals^n$ is a separable operator if
\[
\opT(x)=(\opU_1(x_1),\dots,\opU_m(x_m)),
\]
where $\opU_i\colon\reals^{n_i}\rightarrow\reals^{n_i}$ for $\ieqm$.
Separable operators are coordinate friendly if $\max_{i=1,\dots,m}\mathcal{F}[x_i\mapsto \opU_i(x_i)]\ll\mathcal{F}[x\mapsto \opT(x)]$.

\vspace{0.2in}
Common example: multiplication by a (block) diagonal matrix.

\vspace{0.2in}
$f\colon\reals^n\rightarrow \overline{\reals}$ is a separable function if
\begin{align*}
  f(x)  = \sum\itom f_i(x_i),
\end{align*}
where $f_i\colon\reals^{n_i}\rightarrow \overline{\reals}$ for $\ieqm$.
If $f$ is separable and differentiable, then $\nabla f$ is separable.
If $f$ is separable and CCP, then $\prox_{f}$ is separable.

\end{frame}




\begin{frame}
\frametitle{Separable operators}
A separable constraint is of the form
\begin{align*}
x_i\in C_i\quad
\text{ for }
  \ieqm.
\end{align*}
Projection onto a separable constraint is separable.

\vspace{0.2in}

Common example: box constraint
\begin{align*}
  %a \le x \le b ~~~\text{or}~~~
  a_i\le x_i \le b_i \quad
\text{ for }\ieqm.
\end{align*}
\end{frame}



\begin{frame}
\frametitle{Extended coordinate-friendly}
$\opT$ is extended coordinate-friendly if there is an auxiliary quantity $y(x)$ such that
\begin{align*}
    \max_{i=1,\dots,m}\cF\left[\{x,y(x)\}\mapsto \{\opT_ix, y(\opT_i x)\}\right] \ll\cF\left[x\mapsto \opT x\right].
\end{align*}

In other words, computing $\opT_i(x)$ is efficient if $y(x)$ is maintained.
\end{frame}


\begin{frame}[plain]
\frametitle{More coordinate notation}
%Partition $x\in \reals^n$ into $m$ (non-overlapping) blocks of size $n_1,\dots,n_m$, so $n=n_1+\dots+n_m$.
%Write $x=(x_1,\dots,x_m)$, so $x_i\in \reals^{n_i}$ for $i=1,\dots,m$.
Use notation $x=(x_1,\dots,x_m)$ with $x_i\in \reals^{n_i}$. For $A\in \reals^{r\times n}$, write
\[
A_{:,i}\in \reals^{r\times n_i}
\]
for the submatrix, i.e.,
\[
A=\begin{bmatrix}
A_{:,1}&\cdots&A_{:,m}
\end{bmatrix}
\]
%Note that $A_{i,:}$ would not make sense.
%Write $A^\intercal_{:,i}=(A_{:,i})^\intercal\in \reals^{n_i\times r}$ for $i=1,\dots,m$.
and
\[
Ax=A_{:,1}x_{1}+\dots+A_{:,m}x_{m}.
\]
When $f$ is differentiable, write
\[
\nabla f(x)=
\begin{bmatrix}
\nabla_1 f(x)\\
\vdots\\
\nabla_m f(x)
\end{bmatrix}.
\]
\end{frame}

\begin{frame}
\frametitle{Example: Gradient descent on least squares}
%\begin{example}
Consider
\begin{align*}
\underset{x\in\reals^n}{\mbox{minimize}}
\quad \frac{1}{2}\|Ax-b\|^2,
\end{align*}
where $A\in \reals^{r\times n}$ and $b\in \reals^r$, and
\[
\opT(x)=x-\alpha A^\intercal(Ax-b).
\]

\vspace{0.2in}

$\opT$ is parallelizable, \emph{not} coordinate friendly, but extended coordinate friendly.
\end{frame}


\begin{frame}
%\frametitle{Example: Gradient descent on least squares}
% parallel implementations corresponding to $A^\intercal(Ax^k-b)$ are more efficient, and we have
Evaluation of $\opT$ costs
\[
\cF[x\mapsto \opT x]=\mathcal{O}(rn).
\]

\vspace{0.2in}
Parallelizable (assuming $p<\min\{r,n\}$):
\begin{align*}
\mathcal{F}_p[x\mapsto \opT x]
&=
\mathcal{F}_p[\{A,x\}\mapsto Ax]
+
\mathcal{F}_p[\{A^\intercal,Ax\}\mapsto A^\intercal(Ax)]\\
&= \order{rn/p}
\end{align*}
%Define $\opT_i$ such that
%\begin{align*}
%  \opT_i x = (\opT x)_i=x_i-\alpha A^\intercal_{:,i}(Ax-b)
%\end{align*}
\medskip\pause

Not coordinate friendly:
\begin{align*}
\cF[x\mapsto \opT_i x]
&=\cF[x\mapsto Ax] + \cF[ Ax\mapsto \opT_i x]\\
&=\order{rn}+\order{rn_i}\\
&=\order{rn}
\end{align*}
\end{frame}


\begin{frame}[plain]
%\frametitle{Example: Gradient descent on least squares}

However, extended coordinate friendly with auxiliary quantity $Ax$:
\[
\mathcal{F}\left[
\{x,Ax\}\mapsto\{\opT _ix,A(\opT _ix)\}
\right]=\mathcal{O}(rn_i)
\]
if we use the formula
\begin{gather*}
A(\opT_ix)
%=Ax+A(\opT _ix-x)
=Ax+A_{:,i}((\opT x)_i-x_i).
\end{gather*}
\medskip\pause

Therefore the C-FPI with $\opT$
\begin{align*}
%i(k)&\sim \text{IID Uniform}\{1,\dots,p\}\\
x^{k+1}_{i(k)}&=x_{i(k)}^k-\alpha A^\intercal_{:,i(k)}(y^k-b)\\
x^{k+1}_{j}&=x_{j}^k\qquad\text{for } j\neq i(k)\\
y^{k+1}&=y^k+A_{:,i(k)}(x^{k+1}_{i(k)}-x_{i(k)}^k)
\end{align*}
costs $\mathcal{O}(rn_{i(k)})$ flops per iteration.
% by maintaining the auxiliary quantity $Ax^k-b$.
($x^{k+1}_{j}=x_{j}^k$ costs no operations.)\\
Initialize $x^0=0$ and $y=Ax^0=0$.
\end{frame}

\begin{frame}
%\frametitle{Example: Gradient descent on least squares}
If $r \le n$ or $r \sim n$, %another approach of 
%precomputing $A^\intercal A$ and $A^\intercal b$ and
using $\opT(x)=x-\alpha ((A^\intercal A)x-A^\intercal b)$ is not effective.

\medskip
Precomputing
\[
\cF[\{A,b\}\mapsto \{A^\intercal A,A^\intercal b\}]=\order{rn^2}
\]
can be prohibitively expensive when $r \ll n$, and
\[
\cF[\{x^k,A^\intercal A,A^\intercal b\}\mapsto x^{k+1}_{i}]=\order{nn_{i}},
\]
is larger than $\order{rn_{i}}$.

\medskip\pause
If $r \gg n$ and precomputing $A^\intercal A$ is feasible, then 
\[
\opT(x) = (\opI - \alpha A^\intercal A)x +\alpha A^\intercal b
\]
is coordinate friendly.
\end{frame}

\section{Examples}

\begin{frame}
\frametitle{Example: Coordinate gradient descent}
Consider 
\begin{align*}
\begin{array}{ll}
\underset{x\in \reals^n}{\mbox{minimize}}
  &f(x),
  \end{array}
\end{align*}
where $f$ is differentiable.
RC-FPI applied to $\opI - \alpha \nabla f$
\begin{align*}
  x^{k+1}_{i(k)} & = x^{k}_{i(k)} - \alpha \nabla_{i(k)}f(x^{k}),
  %x^{k+1}_{j} &= x_j^k\quad\text{for } j\neq i(k),
\end{align*}
is randomized/stochastic coordinate gradient method or randomized/stochastic coordinate gradient descent.
Converges if a minimizer exists, $f$ is $L$-smooth, and $\alpha\in (0,2/L)$.

\end{frame}

\begin{frame}
%\frametitle{Example: Coordinate gradient descent}
In general, $\opI - \alpha \nabla f$ may not be extended coordinate friendly.

\medskip

However, the following machine learning setup is extended coordinate friendly
\[
  f(x)  =\sum_{j=1}^{r} \ell_j(a_j^\intercal x - b_j),
\]
where $a_1,\dots,a_r\in \reals^n$, $b_1,\dots,b_r\in \reals$, and $\ell_1,\dots,\ell_r$ are differentiable CCP functions on $\reals$.
%For simplicity, Consider the case where all blocks are single coordinates, i.e., $m=n$ and $n_1=\dots n_m=1$.
%When $\ell_j(x)=(1/2)x^2$ for $j=1,\dots,r$, the problem reduces to the familiar least-squares problem.
% In a more specific example, if we set $\ell_j\equiv\frac{1}{2}|\cdot|^2$, then we obtain the least-squares cost $\frac{1}{2}\|Ax-b\|_2^2$. Let us consider general differentiable $\ell_1,\dots,\ell_m$.
\vspace{0.2in}

Write
\begin{align*}
  A = \begin{bmatrix}
        \text{---}~ a_1^\intercal ~\text{---}\\
         \vdots \\
        \text{---}~ a_r^\intercal ~\text{---}
       \end{bmatrix}\in\reals^{r\times n},
       \qquad
       \ell(y)=\sum^r_{j=1}\ell_j(y_j).
\end{align*}
Then
\[
\nabla \ell(x)=(\ell'_1(x_1),\dots,\ell'_r(x_r)).
\]
\end{frame}



\begin{frame}
%\frametitle{Example: Coordinate gradient descent}
Randomized coordinate gradient descent with $y^k=Ax^k$
\begin{align*}
  x^{k+1}_{i(k)} & = x^{k}_{i(k)} - \alpha A^\intercal_{:,i(k)}  \nabla \ell(y^k-b)\\
  %Ax^{k+1}-b=Ax^{k}-b+A_{:,i(k)}(x^{k+1}_{i(k)}-x^{k}_{i(k)}).
y^{k+1}&=y^k+A_{:,i(k)}(x^{k+1}_{i(k)}-x^{k}_{i(k)})
\end{align*}
has cost per iteration of $\order{rn_{i(k)}}$, if $\max_{j=1,\dots,r}\mathcal{F}[x\mapsto \ell'_j(x)]=\mathcal{O}(1)$.

\vspace{0.2in}

Initialize $x^0=0$ and $y=Ax^0=0$.
\end{frame}


\begin{frame}[fragile,plain]
\frametitle{Example: Coordinate GD with block-wise stepsize}
Consider 
\begin{align*}
\begin{array}{ll}
\underset{x\in \reals^n}{\mbox{minimize}}
  &f(x),
  \end{array}
\end{align*}
where $f$ is $L$-smooth.
For any diagonal matrix
\begingroup\makeatletter\def\f@size{8}\check@mathfonts
\[
D=\begin{bmatrix}
\beta_1I_{n_1}&&&&\\
%&\beta_2I_{n_2}&&&\\
&&\ddots\\
&&&&\beta_mI_{n_m}\\
\end{bmatrix}
\]
\endgroup
where  $\beta_i>0$ and $I_{n_i}$ is the $n_i\times n_i$ identity matrix, the problem is equivalent to
\vspace{-10pt}
\begin{align*}
\begin{array}{ll}
\underset{x\in \reals^n}{\mbox{minimize}}
  &f(Dx).
  \end{array}
\end{align*}
Randomized coordinate gradient method on equivalent problem is
\begin{align*}
  x^{k+1}_{i(k)} & = x^{k}_{i(k)} - \alpha_{i(k)} \nabla_{i(k)}f(x^{k}),
  %x^{k+1}_{j} &= x_j^k\quad\text{for } j\neq i(k),
\end{align*}
where $\alpha_{i(k)}=\alpha \beta_{i(k)}$.
%The method converges of a minimizer exists and $f_i$ is $L$-smooth and $\alpha_i\in (0,2/L_i)$ for $i=1,\dots,m$.
Non-uniform block-wise stepsize is often necessary for a speedup compared to the (full deterministic) gradient method.
\end{frame}

\begin{frame}
\frametitle{Example: Coordinate proximal-gradient descent}
Consider 
\[
% \label{eq:f_sepg}
\begin{array}{ll}
\underset{x\in \reals^n}{\mbox{minimize}}& \displaystyle{f(x)+\sum_{i=1}^{m} g_i(x_i)}
\end{array}
\]
where $f$ is differentiable. So we minimize sum of a differentiable function and a separable function.
% We often use $g_i$ as constraints or regularization on $x_i$. When we have constraints $\ell_i \le x_i\le u_i$, where $\ell_i\in[-\infty,\infty)$ and $u_i\in(-\infty,\infty]$, we let $g_i = \delta_{[\ell_i,u_i]}$.
% XXX The above discussion belongs to Ch2. XXX
Write
\begin{align*}
  g(x) = \sum_{i=1}^{m} g_i(x_i).
\end{align*}
%Since $g$ is separable, so is $\prox_{\alpha g}$.

% \begin{align*}
%   \big(\prox_{\alpha g}(y))_i & = \prox_{\alpha_i g_i}(y_i),\quad i = 1,\dots, m.
% \end{align*}
\medskip\pause

RC-FPI with FBS operator $\prox_{\alpha g}(I-\alpha\nabla f)$
%Therefore, $T_i x = \prox_{\alpha_i g_i}(x_i-\alpha_i\nabla_i f(x))$, and C-FPI gives us the \emph{}:
\begin{align*}
  x^{k+1}_{i(k)} & = \prox_{\alpha g_{i(k)}}\big(x^{k}_{i(k)} - \alpha \nabla_{i(k)}f(x^{k})\big),
%  x^{k+1}_{j} &= x_j^k\quad\text{for } j\neq i(k).
\end{align*}
is coordinate proximal-gradient (descent) method.
Converges if a minimizer exists, $f$ is $L$-smooth, and $\alpha\in(0,2/L)$.
\end{frame}


\begin{frame}
\frametitle{Example: Coordinate proximal-gradient descent}

With block-wise argument, we get
\begin{align*}
  x^{k+1}_{i(k)} & = \prox_{\alpha_{i(k)}g_{i(k)}}\big(x^{k}_{i(k)} - \alpha_{i(k)} \nabla_{i(k)}f(x^{k})\big).
%  x^{k+1}_{j} &= x_j^k\quad\text{for } j\neq i(k).
\end{align*}
Non-uniform block-wise stepsizes important for speedup.


\vspace{0.2in}

When $g$ is not separable, $\prox_{\alpha g}(I-\alpha\nabla f)$ is in general not extended coordinate friendly and RC-FPI not efficient.

%The evaluation of even a single coordinate of $\prox_{\alpha g}$ requires the full output of $x-\alpha \nabla f(x)$.
\end{frame}


\begin{frame}
\frametitle{Example: Stochastic dual coordinate ascent}
Consider 
\begin{align*}
\begin{array}{ll}
\underset{x\in \reals^r}{\mbox{minimize}}& \displaystyle{g(x) + \sum_{i=1}^{n} \ell_i(a_i^\intercal x - b_i)},
  \end{array}
\end{align*}
where $g$ is a strongly convex CCP function on $\reals^r$ (so $g^*$ is smooth)
and $\ell_i$ is a CCP function on $\reals$.
% for $i=1,\dots,n$.
Write
\begin{align*}
  A = \begin{bmatrix}
        \text{---}~ a_1^\intercal ~\text{---}\\
         \vdots \\
        \text{---}~ a_n^\intercal ~\text{---}
       \end{bmatrix} \in\reals^{n\times r},
       \qquad
       b= \begin{bmatrix}
        b_1\\\vdots\\b_n
       \end{bmatrix} \in\reals^{n}.
%   \quad
%   A^\intercaly = \sum_{i=1}^{p}y_j
%   \begin{bmatrix}
%   \vert\\
%   a_j \\
%   \vert\end{bmatrix}.
\end{align*}
Primal problem generated bu
\[
\lagrange(x,u)=g(x)+\langle u,Ax-b\rangle -\sum_{i=1}^{n}\ell_i^*(u_i)
\]
and corresponding dual problem is
\begin{align*}
\begin{array}{ll}
\underset{u\in \reals^n}{\mbox{maximize}}& -g^*\left(-A^\intercal u\right)-b^\intercal u -\sum_{i=1}^{n}\ell_i^*(u_i).
    \end{array}
\end{align*}
\end{frame}


\begin{frame}
\frametitle{Example: Stochastic dual coordinate ascent}
%\begin{align*}
%  u^{k+1}_{i(k)} & = \prox_{\alpha_{i(k)}\ell^*_{i(k)}}\left(u^{k}_{i(k)} + \alpha_{i(k)} \big(A_{i(k),:} \nabla g^*(-A^\intercal u^k)-b_{i(k)}\big)\right),
%\end{align*}
Randomized coordinate proximal-gradient applied to dual
\begin{align*}
  u^{k+1}_{i(k)} & = \prox_{\alpha_{i(k)}\ell^*_{i(k)}}\left(u^{k}_{i(k)} + \alpha_{i(k)} \left(A_{i(k),:} \nabla g^*(y^k)-b_{i(k)}\right)\right)\\
  y^{k+1}&=y^k-A^\intercal_{i(k),:} (u^{k+1}_{i(k)}-u^k_{i(k)})
\end{align*}
is a variation of stochastic dual coordinate ascent.
Assume $\cF[y\mapsto\nabla g^*(y)]=\cO(r)$ and $\max_{i=1,\dots,n}\mathcal{F}[u\mapsto \prox_{\alpha_i \ell^*_i}(u)]=\mathcal{O}(1)$.
Extended coordinate friendly with $y^k=-A^\intercal u^k$ maintained.
We have
\[
\cF[\{y^k,u^k\}\mapsto \{y^{k+1},u^{k+1}\}]=\order{rn_{i(k)}}.
\]
(One can recover the primal solution with $\nabla g^*(y^k)$.)
\end{frame}

\begin{frame}
\frametitle{Note on splitting data}

Iteration of primal coordinate GD accesses $A_{:,i(k)}$, a block of columns.\\
Iteration of dual coordinate GD accesses $A_{i(k),:}$, a block of rows.

\vspace{0.2in}

In machine learning, a row of $A$ is a training sample, and we may not want to split it into parts. In so, dual approach is preferred.

\end{frame}


\begin{frame}[plain]
\frametitle{Example: MISO/Finito}
Consider 
\begin{align*}
\underset{x\in\reals^n}{\mbox{minimize}}\quad r(x) + \frac{1}{m}\sum_{i=1}^{m}f_i(x) ,
\end{align*}
where $f_1,\dots,f_m$ are differentiable.
Use consensus technique to get
\begin{align*}
\underset{\vx\in \reals^{nm}}{\mbox{minimize}}\quad \delta_C(\vx)+\sum_{i=1}^{m}\big(r(x_i)+f_i(x_i)\big),
\end{align*}
where $\vx=(x_1,\dots,x_m)$ and $C$ is the consensus set.
%To clarify, $x_1,\dots,x_m\in \reals^n$.
\vspace{0.2in}

Write $f(\vx)  =\sum_{i=1}^{m} f_i(x_i) $ and $g(\vx)  = \delta_C(\vx)+\sum_{i=1}^{m} r(x_i)$, so
\begin{align*}
  \prox_{\alpha g}(y_1,\dots,y_m) & =  (x,\dots,x),\quad x= \prox_{\alpha r}\left(\frac{1}{m}\sum_{i=1}^{m}y_i\right).
\end{align*}
(See Exercise 2.28.)
\end{frame}

\begin{frame}
% [plain]
% \frametitle{Example: MISO/Finito}
Both FBS and BFS are extended coordinate friendly with $\overline{z}^k$ maintained.\\
\pause
RC-FPI with BFS operator $(\opI-\alpha \nabla f)\prox_{\alpha g}$ is
\begin{align*}
x^k&= \prox_{\alpha r}\Big(\overline{z}^k\Big)\\
  %z_{i(k)}^{k+1} & = z^k_{i(k)} - \alpha \nabla f_{i(k)}(x^k),
  z_{i(k)}^{k+1} & = x^k - \alpha \nabla f_{i(k)}(x^k)\\
  \overline{z}^{k+1} & = \overline{z}^k + \frac{1}{m}  \left(z_{i(k)}^{k+1} - z_{i(k)}^k\right).
 % y^{k+1}_{j} &= y_j^k\quad\text{for } j\neq i(k).
\end{align*}
RC-FPI with FBS operator $\prox_{\alpha g}(\opI-\alpha \nabla f)$ is
\begin{align*}
%z^{k+1}&=x_{i(k)}^k-\alpha \nabla f_{i(k)}(x_{i(k)}^k)\\
x_{i(k)}^{k+1}&=\prox_{\alpha r}(\overline{z}^{k})\\
\overline{z}^{k+1}&=\overline{z}^k+\frac{1}{m}\left(x^{k+1}_{i(k)}-x^k_{i(k)}-\alpha (\nabla f_i(x_{i(k)}^{k+1})-\nabla f_i(x_{i(k)}^k))\right),
\end{align*}
where
$\overline{z}^{k}=\frac{1}{m}\sum^m_{i=1}(x^k_i-\alpha \nabla f_{i(k)}(x_{i}^k))$ is maintained.

\pause
These two equivalent methods are called minimization by incremental surrogate optimization (MISO) or Finito. 
Converges if a solution exists and $\alpha\in(0,2/L)$.
\end{frame}

\begin{frame}
% \frametitle{Example: MISO/Finito}
Among the two, BFS has a minor and subtle advantage.

\vspace{0.2in}
For BFS, one can use $(z^0_1,\dots,z^0_m)=(0,\dots,0)$ and $\overline{z}^0=0$ as the starting point.

\vspace{0.2in}
For FBS, the starting point $(x^0_1,\dots,x^0_m)\in\reals^{nm}$ can be arbitrary, but
\[
\overline{z}^0=\frac{1}{m}\sum^m_{i=1}\left(x^0_i-\alpha\nabla f_i(x^0_i)\right)
\]
needs to be computed before starting the iterations in order to establish convergence via Theorem~2.
%(The BFS version has a minor and subtle advantage.
%For the BFS version, 
%where the starting point $(z^0_1,\dots,z^0,m)\in\reals^{nm}$ is arbitrary.
%where the starting point $(x^0_1,\dots,x^0,m)\in\reals^{nm}$ is arbitrary, but $\bar{z}^0$ 
\end{frame}

\begin{frame}
\frametitle{Example: Conic programs with many small cones}
Consider 
\begin{align*}
\begin{array}{ll}
\underset{x\in\reals^n}{\mbox{minimize}} & c^\intercal x \\
\mbox{subject to}& Ax = b\\
  & x\in Q_1\times\dots\times Q_m,
  \end{array}
\end{align*}
where $Q_i\subseteq \reals^{n_i}$ is a nonempty closed convex set,
% for $i=1,\dots,m$,
$A\in \reals^{r\times n}$ has rank $r$, and $b\in\reals^r$.
Assume $\cF[x_i\mapsto \proj_{Q_{i}}x_i]=C_i$.
\vspace{0.2in}

Note: [$x\in Q_1\times\dots\times Q_m$] $\Leftrightarrow$ [$x_i\in Q_i$ for $i=1,\dots,m$]
\vspace{0.2in}

When $Q_1,\dots,Q_m$ are convex cones, problem called a conic program.

\end{frame}

\begin{frame}
%[plain]
%\frametitle{Example: Conic programs with many small cones}
Naive RC-FPI with DRS applied to 
\begin{align*}
\begin{array}{ll}
\underset{x\in\reals^n}{\mbox{minimize}} & 
\underbrace{c^\intercal x + \delta_{\{x\,|\,Ax=b\}}(x)}_{=f(x)}+
\underbrace{\delta_{ Q_1\times\dots\times Q_m}(x)}_{=g(x)}
  \end{array}
\end{align*}
\vspace{-0.3in}

becomes
\begin{align*}
    x^{k+1/2}_{i}&=\proj_{Q_{i}}(z^k_{i})\qquad\text{for }i=1,\dots,m\\
    z^{k+1}_{i(k)}&=z^k_{i(k)}+D_{i(k),:}(2x^{k+1/2}-z^k)+v_{i(k)}-x^{k+1/2}_{i(k)},\end{align*}
where $D=I-A^\intercal(AA^\intercal)^{-1}A$ and $v=A^\intercal(AA^\intercal)^{-1}b-\alpha Dc$.
(Exercise 2.24.)
Costs $\order{C_1+\dots+C_n+nn_{i(k)}}$ per iteration.

\end{frame}

\begin{frame}

Utilize the extended coordinate friendly structure with $y^k=D2x^{k+1/2}-z^k$:
\begin{align*}
    x^{k+1/2}_{i(k)}&=\proj_{Q_{i(k)}}(z^k_{i(k)})\\
    z^{k+1}_{i(k)}&=z^k_{i(k)}+y^k_{i(k)}+v_{i(k)}-x^{k+1/2}_{i(k)}\\
     y^{k+1}&=D_{:,i(k)}\left(2\proj_{Q_{i(k)}}(z^{k+1}_{i(k)})-2x^{k+1/2}_{i(k)}-z^{k+1}_{i(k)}+z^{k}_{i(k)}\right),
\end{align*}
which costs $\order{C_{i(k)}+nn_{i(k)}}$ per iteration.
\end{frame}

\begin{frame}
\frametitle{Example: Conic programs with large second-order cone}
Second-order cone programming (SOCP) is a generalization linear programming with constraints in the the second-order cone
\[
Q
=
\left\{
x\in \reals^{n}\,\big|\,
\sqrt{x_1^2+\dots+x_{n-1}^2}\le x_n
\right\}.
\]
%\vspace{0.2in}

Write $x=(u,t)$ where $u\in \reals^{n-1}$ and $t\in\reals$.
Projection $x=(u,t)$ to $Q$ scales $u$ and $t$:
\[
\proj_{Q}(x)=(\rho u,\sigma t),\qquad
\left(\rho,\sigma\right)  =\begin{cases}
                               (0,0) & \mbox{if } t \le -\|u\| \\
                               (1,1) & \mbox{if }   \|u\|\le t\\
                         \frac{1}{2}(\|u\|+t)\left(\frac{1}{\|u\|},\frac{1}{t}\right) & \mbox{otherwise}.
                \end{cases}
\]
See Exercise 5.2.
\end{frame}


\begin{frame}
\frametitle{Conic programs with large second-order cone}
Consider 
\begin{align*}
\begin{array}{ll}
\underset{x\in\reals^n}{\mbox{minimize}} & c^\intercal x \\
\mbox{subject to}& Ax = b\\
  & x\in Q
  \end{array}
\end{align*}
where $Q$ is the second-order cone, $A\in \reals^{r\times n}$ with rank $r$, and $b\in\reals^r$.
For simplicity, assume all blocks are single coordinates, i.e., $m=n$ and $n_1=\dots n_m=1$.
\medskip

Apply DRS to
\begin{align*}
\begin{array}{ll}
\underset{x\in\reals^n}{\mbox{minimize}} & 
\underbrace{c^\intercal x + \delta_{\{x\,|\,Ax=b\}}(x)}_{=f(x)}+
\underbrace{\delta_{ Q}(x)}_{=g(x)}.
  \end{array}
\end{align*}

\end{frame}

\begin{frame}
Define $D=I-A^\intercal(AA^\intercal)^{-1}A$ and $v=A^\intercal(AA^\intercal)^{-1}b-\alpha Dc$ %as in Exercise~\ref{exercise:conic_program} 
and write
\[
z^k=(u^k,t^k),
\qquad
D=
\begin{bmatrix}
G&f
\end{bmatrix},
\]
where $G\in \reals^{n\times (n-1)}$ and $f\in \reals^{n}$.
The DRS operator is extended coordinate friendly, and we can implement the RC-FPI with DRS as
\begin{align*}
    x^{k+1/2}_{i(k)}&=\proj_{Q}(z^k_{i(k)})\\
    z^{k+1}_{i(k)}
    &=z^k_{i(k)}+v_{i(k)}-x^{k+1/2}_{i(k)}\\
    &\quad +(2\rho(\|u^k\|,t^k)-1)(Gu^k)_{i(k)}+(2\sigma(\|u^k\|,t^k)-1)t^kf_{i(k)}\\
    \|u^{k+1}\|^2&=
\left\{
\begin{array}{ll}
\|u^k\|^2+(z^{k+1}_{i(k)})^2-(z^k_{i(k)})^2&\text{ if }i(k)<n\\
\|u^k\|^2&\text{ if }i(k)=n
\end{array}
\right.\\
Gu^{k+1}&=
\left\{
\begin{array}{ll}
Gu^k+G_{:,i(k)}(u^{k+1}_{i(k)}-u^{k}_{i(k)})&\text{ if }i(k)<n\\
Gu^k&\text{ if }i(k)=n.
\end{array}
\right.
\end{align*}
This implementation costs $\order{n}$ flops per iteration.

\end{frame}

\section{Discussion}

\begin{frame}{Discussion}
\begin{itemize}
    \item RC-FPI may provide a speedup but no guarantees.
    \bigskip\pause
    
    \item Larger stepsizes often, but not always, translate to speedup.
    \bigskip\pause
    
    \item RC-FPI is a precursor to asynchronous FPI in Chapter 6.
\end{itemize}    
\end{frame}

\iffalse
\begin{frame}
\frametitle{Discussion}
RC-FPI may provide a speedup over FPI when the operator is extended coordinate friendly
and very likely when coordinate-wise stepsizes are larger than the step size used by FPI.

\vspace{0.2in}

When comparing RC-FPI and FPI, it is useful to compare 1 iteration of FPI with 1 epoch (defined as $m$ iterations) of RC-FPI. Given (extended) coordinate friendly, 1 epoch of RC-FPI and 1 iteration of FPI have similar computational costs.

\vspace{0.2in}
Theorems~1 and 2 guarantee a similar amount of reduction in fixed-point residual with 1 iteration of FPI and with 1 epoch of RC-FPI.
%However, this does not necessarily mean one iteration of FPI and one epoch of RC-FPI actually makes the same amount of progress.
In practice, RC-FPI and FPI often converge much faster than what Theorems~1 and 2 guarantee.
If so, the similarity in the guarantees does not have much bearing on the similarity or difference of the actual performances.
\end{frame}

\begin{frame}
The ability to use larger stepsizes often, but not always, translates to a speedup.
There are cases where the stepsize limitation is the bottleneck of the algorithm and alleviating it leads to a speedup.
In some cases, there is theoretical support for this observation.
For example, an epoch of the randomized coordinate gradient method achieves a greater reduction in function value than an iteration of the (deterministic full) gradient method when different stepsizes are used for the different blocks.
Such analyses directly utilizes the subgradient inequality, rather than the resulting monotonicity inequality.
%the convexity of the functions and therefore goes beyond the scope of this book.
\medskip

% i.e., in some cases, an epoch of RC-FPI provably makes more progress than an iteration of FPI.


RC-FPI may offer no speed-up. There are empirical examples where epochs of RC-FPI and iterations of FPI converge with the same rate. We are not aware of an cases where epochs of RC-FPI make less progress than iterations of FPI.
%(Even though both methods are much faster than the theoretical guarantees provided by Theorems~\ref{thm:averaged-iter} and \ref{thm:stochastic-iter}.)
\medskip


Nevertheless, studying RC-FPI without any guarantee of speedup serves as a precursor to the asynchronous FPI we discuss in Chapter 6, which achieves more per unit time.
%In parallel computing, asynchrony can increase the number of iterations run per unit time.
%Even if an epoch of asynchronous FPI makes the same amount of progress as an iteration of FPI, asynchronuous FPI can make more progress per unit time.
\end{frame}


% \begin{frame}
% \frametitle{XXX}
% In this chapter, we present the stochastic coordinate-update fixed-point iteration (RC-FPI), a randomized method that updates a randomly chosen coordinate.
% \end{frame}

\begin{frame}
Most conic programs in practice involve multiple cones, rather than a single large second-order cone.
The previous example demonstrates that multiple cones can be handled separately.
The current example demonstrates that a single large second-order cone can be furthermore split down to the individual coordinates.
When a conic program is given by a mixture of small cones and large second-order cones, we can combine these two approaches to implement an extended coordinate friendly RC-FPI.
Note that the nonnegative orthant cone
\[
\reals^n_+=\{(x_1,\dots,x_n)\in \reals^n\,|\,x_1,\dots,x_n\ge 0\},
\]
used in linear programming, already splits into $\reals_+\times\cdots\times\reals_+$.
Large semi-definite cones, used in semidefinite programming, do not split in a similar manner.
\end{frame}

\fi



\end{document}
