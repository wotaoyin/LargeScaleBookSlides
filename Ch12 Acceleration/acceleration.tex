\documentclass[10pt,mathserif]{beamer}

\usepackage{graphicx,amsmath,amssymb,psfrag,mathtools}
\usepackage{soul}
\usepackage{amsmath,amsfonts,amsthm,bbm}
\usepackage{stmaryrd}
\usepackage{subcaption}



%Code block environment
\usepackage{listings}


\definecolor{lightgrey}{gray}{0.8}
\definecolor{medgrey}{gray}{0.6}
\definecolor{darkgrey}{gray}{0.4}
\usepackage{xcolor}
\lstset { %
    backgroundcolor=\color{black!5}, % set backgroundcolor
    basicstyle=\ttfamily,
    showstringspaces=false,
    commentstyle = \ttfamily,
    commentstyle=\color{commentgreen}\ttfamily,
    morecomment=[l][\color{darkgrey}]{//},
}



\usepackage{tikz}
\usetikzlibrary{matrix,chains,positioning,decorations.pathreplacing,arrows}
\usetikzlibrary{positioning,calc}
\usepackage{tkz-euclide}
%\usetkzobj{all}





%-------------------------------------------------------------------------------
%Definition of operator font
 \usepackage[bb=boondox]{mathalfa}
%% import \varmathbb without affecting other fonts
\usepackage{xparse}
\DeclareFontFamily{U}{ntxmia}{}
\DeclareFontShape{U}{ntxmia}{m}{it}{<-> ntxmia }{}
\DeclareFontShape{U}{ntxmia}{b}{it}{<-> ntxbmia }{}
\DeclareSymbolFont{lettersA}{U}{ntxmia}{m}{it}
\SetSymbolFont{lettersA}{bold}{U}{ntxmia}{b}{it}
\ExplSyntaxOn
\NewDocumentCommand{\varmathbb}{m}
 {
  \tl_map_inline:nn { #1 }
   {
    \use:c { varbb##1 }
   }
 }
\tl_map_inline:nn { ABCDEFGHIJKLMNOPQRSTUVWXYZ }
 {
  \exp_args:Nc \DeclareMathSymbol{varbb#1}{\mathord}{lettersA}{\int_eval:n { `#1+67 }}
 }
\exp_args:Nc \DeclareMathSymbol{varbbk}{\mathord}{lettersA}{169}
\ExplSyntaxOff
%%
\makeatletter
\DeclareFontFamily{U}{tipa}{}
\DeclareFontShape{U}{tipa}{m}{n}{<->tipa10}{}
\newcommand{\arc@char}{{\usefont{U}{tipa}{m}{n}\symbol{62}}}%

\newcommand{\arc}[1]{\mathpalette\arc@arc{#1}}

\newcommand{\arc@arc}[2]{%
  \sbox0{$\m@th#1#2$}%
  \vbox{
    \hbox{\resizebox{\wd0}{\height}{\arc@char}}
    \nointerlineskip
    \box0
  }%
}
\makeatother
\newcommand{\opA}{{\varmathbb{A}}}
\newcommand{\opB}{{\varmathbb{B}}}
\newcommand{\opC}{{\varmathbb{C}}}
\newcommand{\opD}{{\varmathbb{D}}}
\newcommand{\opE}{{\varmathbb{E}}}
\newcommand{\opF}{{\varmathbb{F}}}
\newcommand{\opG}{{\varmathbb{G}}}
\newcommand{\opH}{{\varmathbb{H}}}
\newcommand{\opI}{{\varmathbb{I}}}
\newcommand{\opJ}{{\varmathbb{J}}}
\newcommand{\opK}{{\varmathbb{K}}}
\newcommand{\opL}{{\varmathbb{L}}}
\newcommand{\opM}{{\varmathbb{M}}}
\newcommand{\opN}{{\varmathbb{N}}}
\newcommand{\opO}{{\varmathbb{O}}}
\newcommand{\opP}{{\varmathbb{P}}}
\newcommand{\opQ}{{\varmathbb{Q}}}
\newcommand{\opR}{{\varmathbb{R}}}
\newcommand{\opS}{{\varmathbb{S}}}
\newcommand{\opT}{{\varmathbb{T}}}
\newcommand{\opU}{{\varmathbb{U}}}
\newcommand{\opV}{{\varmathbb{V}}}
\newcommand{\opW}{{\varmathbb{W}}}
\newcommand{\opX}{{\varmathbb{X}}}
\newcommand{\opY}{{\varmathbb{Y}}}
\newcommand{\opZ}{{\varmathbb{Z}}}
\newcommand{\opZer}{\mathbb{0}}
%-------------------------------------------------------------------------------



%-------------------------------------------------------------------------------
%Definition of other font types
\newcommand{\va}{{\mathbf{a}}}
\newcommand{\vb}{{\mathbf{b}}}
\newcommand{\vc}{{\mathbf{c}}}
\newcommand{\vd}{{\mathbf{d}}}
\newcommand{\ve}{{\mathbf{e}}}
\newcommand{\vf}{{\mathbf{f}}}
\newcommand{\vg}{{\mathbf{g}}}
\newcommand{\vh}{{\mathbf{h}}}
\newcommand{\vi}{{\mathbf{i}}}
\newcommand{\vj}{{\mathbf{j}}}
\newcommand{\vk}{{\mathbf{k}}}
\newcommand{\vl}{{\mathbf{l}}}
\newcommand{\vm}{{\mathbf{m}}}
\newcommand{\vn}{{\mathbf{n}}}
\newcommand{\vo}{{\mathbf{o}}}
\newcommand{\vp}{{\mathbf{p}}}
\newcommand{\vq}{{\mathbf{q}}}
\newcommand{\vr}{{\mathbf{r}}}
\newcommand{\vs}{{\mathbf{s}}}
\newcommand{\vt}{{\mathbf{t}}}
\newcommand{\vu}{{\mathbf{u}}}
\newcommand{\vv}{{\mathbf{v}}}
\newcommand{\vw}{{\mathbf{w}}}
\newcommand{\vx}{{\mathbf{x}}}
\newcommand{\vy}{{\mathbf{y}}}
\newcommand{\vz}{{\mathbf{z}}}

\newcommand{\vA}{{\mathbf{A}}}
\newcommand{\vB}{{\mathbf{B}}}
\newcommand{\vC}{{\mathbf{C}}}
\newcommand{\vD}{{\mathbf{D}}}
\newcommand{\vE}{{\mathbf{E}}}
\newcommand{\vF}{{\mathbf{F}}}
\newcommand{\vG}{{\mathbf{G}}}
\newcommand{\vH}{{\mathbf{H}}}
\newcommand{\vI}{{\mathbf{I}}}
\newcommand{\vJ}{{\mathbf{J}}}
\newcommand{\vK}{{\mathbf{K}}}
\newcommand{\vL}{{\mathbf{L}}}
\newcommand{\vM}{{\mathbf{M}}}
\newcommand{\vN}{{\mathbf{N}}}
\newcommand{\vO}{{\mathbf{O}}}
\newcommand{\vP}{{\mathbf{P}}}
\newcommand{\vQ}{{\mathbf{Q}}}
\newcommand{\vR}{{\mathbf{R}}}
\newcommand{\vS}{{\mathbf{S}}}
\newcommand{\vT}{{\mathbf{T}}}
\newcommand{\vU}{{\mathbf{U}}}
\newcommand{\vV}{{\mathbf{V}}}
\newcommand{\vW}{{\mathbf{W}}}
\newcommand{\vX}{{\mathbf{X}}}
\newcommand{\vY}{{\mathbf{Y}}}
\newcommand{\vZ}{{\mathbf{Z}}}

\newcommand{\cA}{{\mathcal{A}}}
\newcommand{\cB}{{\mathcal{B}}}
\newcommand{\cC}{{\mathcal{C}}}
\newcommand{\cD}{{\mathcal{D}}}
\newcommand{\cE}{{\mathcal{E}}}
\newcommand{\cF}{{\mathcal{F}}}
\newcommand{\cG}{{\mathcal{G}}}
\newcommand{\cH}{{\mathcal{H}}}
\newcommand{\cI}{{\mathcal{I}}}
\newcommand{\cJ}{{\mathcal{J}}}
\newcommand{\cK}{{\mathcal{K}}}
\newcommand{\cL}{{\mathcal{L}}}
\newcommand{\cM}{{\mathcal{M}}}
\newcommand{\cN}{{\mathcal{N}}}
\newcommand{\cO}{{\mathcal{O}}}
\newcommand{\cP}{{\mathcal{P}}}
\newcommand{\cQ}{{\mathcal{Q}}}
\newcommand{\cR}{{\mathcal{R}}}
\newcommand{\cS}{{\mathcal{S}}}
\newcommand{\cT}{{\mathcal{T}}}
\newcommand{\cU}{{\mathcal{U}}}
\newcommand{\cV}{{\mathcal{V}}}
\newcommand{\cW}{{\mathcal{W}}}
\newcommand{\cX}{{\mathcal{X}}}
\newcommand{\cY}{{\mathcal{Y}}}
\newcommand{\cZ}{{\mathcal{Z}}}
%-------------------------------------------------------------------------------




%-------------------------------------------------------------------------------
%% macros for math notions and operators
\newcommand{\EE}{{\mathbb{E}}}
\newcommand{\expec}{\mathbb{E}}
\newcommand{\Prob}{{\mathrm{Prob}}} % probability

\newcommand{\reals}{\mathbb{R}}
\newcommand{\RR}{\mathbb{R}}
\newcommand{\complex}{\mathbb{C}}
\newcommand{\CC}{\mathbb{C}}
\newcommand{\nats}{\mathbb{N}}
\newcommand{\NN}{\mathbb{N}}
\newcommand{\ZZ}{\mathbb{Z}}
\newcommand{\bigO}{\mathcal{O}}
\newcommand{\order}[1]{{\mathcal{O}\left(#1\right)}}
\renewcommand{\SS}{{\mathbb{S}}}
\newcommand{\SSp}{\mathbb{S}_{+}}
\newcommand{\SSpp}{\mathbb{S}_{++}}
\newcommand{\sign}{\mathrm{sign}}
\newcommand{\vzero}{\mathbf{0}}
\newcommand{\vone}{{\mathbf{1}}}

\renewcommand{\Re}{\operatorname{Re}} 	%Real part
\renewcommand{\Im}{\operatorname{Im}}	%imaginary part

%\newcommand{\supp}{{\mathrm{supp}}} % support
\newcommand{\range}{\mathrm{range}\,} % domain
\newcommand{\tr}{{\mathrm{tr}}} % trace
%-------------------------------------------------------------------------------





%-------------------------------------------------------------------------------
%% Theorem definitions
\setbeamertemplate{theorems}[ams style] 
\newtheorem*{theorem*}{Theorem}
%\newtheorem{lemma}{Lemma}    % already provided by amsthm
\newtheorem{proposition}{Proposition}
%\newtheorem{proof}{Proof}  % already provided by amsthm


%-------------------------------------------------------------------------------
%% operator and convex analysis definitions

\newcommand*{\fix}{\mathrm{Fix}\,}
\newcommand*{\zer}{\mathrm{Zer}\,}
\newcommand*{\gra}{\mathrm{Gra}\,}
\newcommand{\prox}{\mathrm{Prox}}
\newcommand{\proj}{\Pi}
\newcommand{\aff}{\mathrm{aff}\,}    %affine hull
\newcommand{\intr}{\mathrm{int}\,}   %interior
\newcommand{\relint}{\mathrm{ri}\,}  %relative interior
\newcommand{\dom}{\mathrm{dom}\,} % domain
\newcommand{\epi}{\mathrm{epi}\,} % epigraph
\newcommand{\dist}{\mathrm{dist}}
\newcommand{\lagrange}{\mathbf{L}}  %saddle function
\newcommand{\fitzpatrick}{\mathbf{F}}   %Fitzpatrick function
\newcommand{\vecdelay}{\boldsymbol{d}}   %vector delay
\DeclareMathOperator*{\argmin}{argmin}
\DeclareMathOperator*{\argmax}{argmax}

%-------------------------------------------------------------------------------
%SRG definitions
\newcommand{\ereal}{\overline{\mathbb{R}^2}}
\newcommand{\ecomplex}{\overline{\mathbb{C}}}
\newcommand{\binfty}{{\boldsymbol \infty}}
\newcommand{\rarc}{\mathrm{Arc}^+}
\newcommand{\larc}{\mathrm{Arc}^-}


%-------------------------------------------------------------------------------




%-------------------------------------------------------------------------------
%Miscellaneous Stuff
%% sequences
\newcommand{\itom}{_{i=1}^{m}}
\newcommand{\ieqm}{i=1,\dots,m}

% use \numberthis to add an equation number in align*
\newcommand\numberthis{\addtocounter{equation}{1}\tag{\theequation}}

\newcolumntype{P}[1]{>{\centering\arraybackslash}p{#1}}

\mode<presentation>
{
\usetheme{default}
}
\setbeamertemplate{navigation symbols}{}
\usecolortheme[rgb={0.13,0.28,0.59}]{structure}
\setbeamertemplate{itemize subitem}{--}
\setbeamertemplate{frametitle} {
	\begin{center}
	  {\large\bf \insertframetitle}
	\end{center}
}

\newcommand\footlineon{
  \setbeamertemplate{footline} {
    \begin{beamercolorbox}[ht=2.5ex,dp=1.125ex,leftskip=.8cm,rightskip=.6cm]{structure}
      \footnotesize \insertsection
      \hfill
      {\insertframenumber}
    \end{beamercolorbox}
    \vskip 0.45cm
  }
}
\footlineon


\newcommand\footlineoff{
  \setbeamertemplate{footline} {
    \begin{beamercolorbox}[ht=2.5ex,dp=1.125ex,leftskip=.8cm,rightskip=.6cm]{structure}
      \footnotesize 
      \hfill
      {\insertframenumber}
    \end{beamercolorbox}
    \vskip 0.45cm
  }
}


\newcommand\blfootnote[1]{%
  \begingroup
  \renewcommand\thefootnote{}\footnote{#1}%
  \addtocounter{footnote}{-1}%
  \endgroup
}


\AtBeginSection[] 
{ 
	\begin{frame}<beamer> 
		\frametitle{Outline} 
		\tableofcontents[currentsection,currentsubsection] 
	\end{frame} 
} 

%% wotao's preference

        % itemize, black bullet, %150 spacing between items using "witemize"
        \newenvironment{witemize}{\itemize\addtolength{\itemsep}{0.3\baselineskip}}{\enditemize}

\iffalse
        % \setbeamertemplate{itemize items}[square]
        \setbeamertemplate{itemize items}{\textbullet}
        \setbeamercolor{itemize item}{fg=black}
        \setbeamercolor{itemize subitem}{fg=black}
        \setbeamercolor{itemize subsubitem}{fg=black}
        \setbeamercolor{enumerate item}{fg=black}
        \setbeamercolor{enumerate subitem}{fg=black}
        \setbeamercolor{enumerate subsubitem}{fg=black}
        \setbeamertemplate{itemize/enumerate body begin}{\normalsize}
        \setbeamertemplate{itemize/enumerate subbody begin}{\normalsize}
        \setbeamertemplate{itemize/enumerate subsubbody begin}{\normalsize}

        % itemize enumerate use normal sized texts
        \setbeamertemplate{itemize/enumerate body begin}{\normalsize}
        \setbeamertemplate{itemize/enumerate subbody begin}{\normalsize}
        \setbeamertemplate{itemize/enumerate subsubbody begin}{\normalsize}

        % block, black over gray with no shadow
        \setbeamertemplate{blocks}[rounded][shadow=false]
        \setbeamercolor{block title}{fg=black,bg=gray!40}
        \setbeamercolor{block body}{fg=black,bg=gray!10}
\fi

\author{Ernest K. Ryu and Wotao Yin}

\date{Large-Scale Convex Optimization via Monotone Operators}


\title{\large \bfseries Acceleration}

\begin{document}

\frame{
\thispagestyle{empty}
\titlepage
}


\begin{frame}
\frametitle{Acceleration}
Theorem~1 establishes a $\mathcal{O}(1/k)$ rate on the squared norm of the fixed-point residual and a similar $\mathcal{O}(1/k)$ rate can be established for the other setups.
This rate can be improved (at least in the worst-case rate).

\vspace{0.2in}

In optimization, an acceleration is a modification of a base method that improves the convergence rate.
An improvement from $\mathcal{O}(1/k)$ to $\mathcal{O}(1/k^2)$ is most common for first-order algorithms.

\vspace{0.2in}

Accelerations is an active topic of research. We keep the discussion minimal and discuss Nesterov's AGM and APPM/OHM.
\end{frame}


\begin{frame}
\frametitle{Accelerated gradient method}
Consider 
\begin{align*}
\begin{array}{ll}
\underset{x\in \reals^n}{\mbox{minimize}}
  &f(x),
  \end{array}
\end{align*}
where $f$ is convex and $L$-smooth.
The method
\begin{align*}
x^{k+1}&=y^k-\frac{1}{L} \nabla f(y^k)\\
y^{k+1}&=x^{k+1}+\frac{k-1}{k+2}(x^{k+1}-x^{k}),
\end{align*}
where $x^0=y^0$, is Nesterov's accelerated gradient method (AGM).
\vspace{0.2in}

\setcounter{theorem}{16}
\begin{theorem}
\label{thm:agm}
Assume the convex, $L$-smooth function $f$ has a minimizer $x^\star$.
Then AGM converges with the rate
\[
f(x^k)-f(x^\star)
\le 
\frac{2L\|x^0-x^\star\|^2}{k^2}
\]
for $k=1,2,\dots$.
\end{theorem}
\end{frame}

\begin{frame}
\frametitle{Proof of Theorem~\ref{thm:agm}}
Equivalent form of AGM:
\begin{align*}
x^{k+1}&=y^k-\frac{1}{L} \nabla f(y^k)\\
z^{k+1}&=z^k-\frac{k+1}{2L}\nabla f(y^k)\\
y^{k+1}&=\left(1-\frac{2}{k+2}\right)x^{k+1}+\frac{2}{k+2}z^{k+1},
\end{align*}
where $x^0=y^0=z^0$.
 (cf.\ Exercise~12.1).
\end{frame}


\begin{frame}[plain]
\frametitle{Proof of Theorem~\ref{thm:agm}}
Preliminary observations.
Define
\[
\theta_{k}=\frac{k+1}{2}
\]
for $k=-1,0,1,\dots$. 
It is straightforward to verify
\begin{equation}
\theta_k^2-\theta_k\le \theta_{k-1}^2
\label{eq:agm-phi-rel}
\end{equation}
for $k=0,1,\dots$.
% Define 
% \[
% z^k=\theta_ky^k-\left(\theta_k-1\right)x^k.
% \]
% With basic calculations, we have
% \begin{equation}
% z^{k+1}=z^k+\frac{\theta_k}{L}\nabla f(x^k).    
% \label{eq:agm-z-def}
% \end{equation}
We will use the inequalities
\begin{gather}
f(x^{k+1})
-f(y^{k})+\frac{1}{2L}\|\nabla f(y^k)\|^2
\le 0
\label{eq:agm-ineq1}\\
f(y^{k})-f(x^{k})\le\langle \nabla f(y^k),y^k-x^k\rangle
\label{eq:agm-ineq2}\\
f(y^{k})-f(x^{\star})\le\langle \nabla f(y^k),y^k-x^\star\rangle.
\label{eq:agm-ineq3}
\end{gather}
The first, \eqref{eq:agm-ineq1}, follows from $L$-smoothness, which implies $f(x)-\frac{L}{2}\|x-y^k\|^2$ is concave as a function of $x$, which in turn implies
\[
f(x)-\frac{L}{2}\|x-y^k\|^2\le 
f(y^{k})+\langle \nabla f(y^k),x-y^k\rangle.
\]
We plug in $x=x^{k+1}=y^k-\frac{1}{L}\nabla f(y^k)$ to get \eqref{eq:agm-ineq1}.
The second and third inequalities, \eqref{eq:agm-ineq2} and \eqref{eq:agm-ineq3}, follow from convexity of $f$.
\end{frame}


\begin{frame}
\frametitle{Proof of Theorem~\ref{thm:agm}}
Define
\[
V^k=
\theta_{k-1}^2
\left(f(x^k)-f(x^\star)\right)+\frac{L}{2}\|z^k-x^\star\|^2.
\]
If we establish $V^{k}\le V^{k-1}\le \dots \le V^0$, then $V^k\le V^0$ implies
\[
\theta_{k-1}^2(f(x^k)-f(x^\star))\le V^k\le V^0= \frac{2L}{k^2}\|z^0-x^\star\|^2.
\]
\end{frame}


\begin{frame}[plain,fragile]
\frametitle{Proof of Theorem~\ref{thm:agm}}
\vspace{-0.2in}
\begingroup\makeatletter\def\f@size{9}\check@mathfonts
\begin{align*}
&V^{k+1}-V^k\\
&=
\theta_{k}^2
\left(f(x^{k+1})-f(x^\star)
+\frac{1}{2L}\|\nabla f(y^k)\|^2
\right)
-
\theta_{k-1}^2(f(x^{k})-f(x^\star))\\
&\qquad
-\theta_k\langle \nabla f(y^k),z^k-x^\star\rangle\\
&\stackrel{\eqref{eq:agm-ineq1}}{\le}
\theta_{k}^2
\left(f(y^k)-f(x^\star)
\right)
-
\theta_{k-1}^2(f(x^{k})-f(x^\star))
-\theta_k\langle \nabla f(y^k),z^k-x^\star\rangle\\
&=
(\theta_k^2-\theta_k)(f(y^k)-f(x^k))+\theta_k(f(y^k)-f(x^k))+(\theta_k^2-\theta_{k-1}^2)(f(x^k)-f(x^\star))\\
&\qquad-\theta_k\langle \nabla f(y^k),z^k-x^\star\rangle\\
&\stackrel{\eqref{eq:agm-phi-rel}}{\le}
(\theta_k^2-\theta_k)(f(y^k)-f(x^k))+\theta_k(f(y^k)-f(x^\star))-\theta_k\langle \nabla f(y^k),z^k-x^\star\rangle\\
&\stackrel{\eqref{eq:agm-ineq2}, \eqref{eq:agm-ineq3}}{\le}
(\theta_k^2-\theta_k)\langle \nabla f(y^k),y^k-x^k\rangle
+
\theta_{k}\langle \nabla f(y^k),y^k-x^\star\rangle
-\theta_k\langle \nabla f(y^k),z^k-x^\star\rangle\\
%&=\theta_{k}^2\left< \nabla f(y^k), y^k-\left(1-\frac{1}{\theta_{k}}\right)x^k-\frac{1}{\theta_{k}}z^k\right>\\
&=
\theta_k\langle \nabla f(y^k),(1-\theta_{k})x^k+\theta_ky^k-z^k\rangle
\stackrel{\text{def.\ of } z^{k}}{=}0,
\end{align*}
\endgroup
where the first equality follows from
\begingroup\makeatletter\def\f@size{9}\check@mathfonts
\[
\frac{L}{2}\left\|z^k-x^\star-\frac{\theta_k}{L}\nabla f(y^k)\right\|^2-\frac{L}{2}\|z^k-x^\star\|^2=
-\theta_k\langle \nabla f(y^k),z^k-x^\star\rangle+
\frac{\theta_k^2}{2L}\|\nabla f(y^k)\|^2.
\]
\endgroup

\vspace{-0.20in}
\qed
\end{frame}

\begin{frame}[fragile,plain]
\frametitle{Comparison with gradient descent}
Gradient descent with stepsize $\alpha=1/L$
\begin{align*}
    x^{k+1} = x^k - \frac{1}{L}\nabla f(x^k)
\end{align*}
converges at rate $\mathcal{O}(1/k)$.
To see this, define
\[
V^{k}=k(f(x^k)-f(x^\star))+\frac{L}{2}\|x^k-x^\star\|^2
\]
and note
\begingroup\makeatletter\def\f@size{8}\check@mathfonts
\begin{align*}
&\hspace*{-0.25in}
V^{k+1}-V^k\\
&\hspace*{-0.15in}=(k+1)(f(x^{k+1})-f(x^k))+f(x^k)-f(x^\star)+\frac{L}{2}\left(\frac{1}{L^2}\|\nabla f(x^k)\|^2-\frac{2}{L}\langle \nabla f(x^k),x^k-x^\star\rangle \right)\\
&\hspace*{-0.15in}\le-\frac{k+1}{2L}\|\nabla f(x^k)\|^2+\langle \nabla f(x^k),x^k-x^\star\rangle+\frac{1}{2L}\|\nabla f(x^k)\|^2-\langle \nabla f(x^k),x^k-x^\star\rangle\\
&\hspace*{-0.15in}=-\frac{k}{2L}\|\nabla f(x^k)\|^2\le 0,
\end{align*}
\endgroup
where inequality follows from analogues of \eqref{eq:agm-ineq1} and \eqref{eq:agm-ineq3}.
$V^k\le V^0$ implies
\[
f(x^k)-f(x^\star)\le \frac{L}{2k}\|x^0-x^\star\|^2-\frac{L}{2k}\|x^k-x^\star\|^2 \le  \frac{L}{2k}\|x^0-x^\star\|^2.
\]
\end{frame}


\begin{frame}
\frametitle{Constructing the Lyapunov function}
The non-increasing quantity $V^k$ is called a Lyapunov function, energy function, or potential function, and the style of proof relying on such quantities is called a Lyapunov analysis. 
Convergence proofs based on Lyapunov analyses tend to be more concise.
Constructing a $V^k$ is a highly non-trivial art, and we briefly outline the process for GD and AGM.

\vspace{0.2in}

Imagine analyzing GD, and we suspect the convergence rate is $f(x^k) - f(x^\star) = \mathcal{O}(1/k)$.
We define $W^k=k(f(x^k)-f(x^\star))$ and, through some analysis, find
\[
W^{k+1}-W^k \le
%-\frac{kL}{2}\|x^{k+1}-x^k\|^2 +
\frac{L}{2}\|x^k - x^\star\|^2-\frac{L}{2}\|x^{k+1} - x^\star\|^2.
\]
So we define $V^k=k(f(x^k)-f(x^\star))+\frac{L}{2}\|x^k-x^\star\|^2$ and present a Lyapunov analysis.

% One might be curious to try, instead, $W^k= t_k^2 (f(x^k) - f(x^\star))$ for some  to prove the (untrue) rate of $O(1/k^2)$ for GD.
\end{frame}


\begin{frame}
\frametitle{Constructing the Lyapunov function}

We may try to prove a faster rate for GD by defining $W^k= t_k^2 (f(x^k) - f(x^\star))$ with a yet unspecified $t_k$-sequence and analyzing $W^{k+1}-W^k$.
If $t_k=\mathcal{O}(k)$, then perhaps we can establish an  $O(1/k^2)$ rate.
However, such an effort does not leads nowhere.

\vspace{0.2in}


For AGM, we again define $W^k= t_{k-1}^2 (f(x^k) - f(x^\star))$ and analyze $W^{k+1}-W^k$.
We can show
\[
W^{k+1}-W^k\le \frac{L}{2}\|z^k-x^\star\|^2-\frac{L}{2}\|z^{k+1}-x^\star\|^2.
\]
for $t_{k}^2-t_{k}\le t_{k-1}^2$ and $t_{k} \ge 0$. An admissible sequence is $t_k = (k+1)/2$.
%, which is satisfied by a sequence with a growth rate $t_k=\mathcal{O}(k)$.
So we define $V^k=t_k^2(f(x^k)-f(x^\star))+\frac{L}{2}\|z^k-x^\star\|^2$ and present a Lyapunov analysis.

% for some $t_k=\mathcal{O}(k)$, aiming to proof $O(1/k^2)$.
\end{frame}

\begin{frame}
\frametitle{Accelerated proximal point and optimized Halpern}
Consider
\begin{align*}
\begin{array}{ll}
\underset{x\in \reals^n}{\mbox{find}}
  &0\in \opA x,
  \end{array}
\end{align*}
where $\opA$ is maximal monotone.
The method
\begin{align*}
y^{k+1}&=\opJ_{\opA}x^k\\
x^{k+1}&=y^{k+1}+\frac{k}{k+2}(y^{k+1}-y^k)-\frac{k}{k+2}(y^k-x^{k-1}),
\end{align*}
where $y^0=x^0$, is the accelerated proximal point method (APPM).
\vspace{0.2in}

Also consider 
\begin{align*}
\begin{array}{ll}
\underset{x\in \reals^n}{\mbox{find}}
  &x= \opT x,
  \end{array}
\end{align*}
where $\opT\colon\reals^n\rightarrow\reals^n$ is nonexpansive.
We call
\[
x^{k+1}=\frac{1}{k+2}x^0+\frac{k+1}{k+2}\opT x^k
\]
the optimized Halpern method (OHM).
\end{frame}


\begin{frame}
\frametitle{Accelerated proximal point and optimized Halpern}

With $\opT=\opR_\opA$, finding elements of $\zer \opA$ and $\fix \opT$ are equivalent (cf.\ Exercise~10.1).
The two methods APPM and OHM are equivalent (cf.\ Exercise~12.2).

\begin{theorem}
\label{thm:appm}
Assume the maximal monotone operator $\opA$ has a zero $x^\star$.
Then APPM/OHM converges with the rate
\[
\|x^{k-1}-\opJ_\opA x^{k-1}\|^2\le \frac{\|x^0-x^\star\|^2}{k^2}
\]
for $k=1,2,\dots$.
\end{theorem}
\vspace{0.2in}

We can equivalently state this result as
\[
\|\opT x^{k-1}-x^{k-1}\|^2\le \frac{4\|x^0-x^\star\|^2}{k^2}.
\]
\end{frame}

\begin{frame}
\frametitle{Proof of Theorem~\ref{thm:appm}}
Define $\tilde{\opA}y^k=x^{k-1}-y^k$, which implies $\tilde{\opA}y^k\in \opA y^k$.
Define
\[
V^k=k^2\|\tilde{\opA}y^k\|^2+k\langle \tilde{\opA}y^k,y^k-x^0\rangle
\]
for $k=0,1,\dots$.

\vspace{0.2in}

Then
\[
V^{k+1}- V^k=-k(k+1)\langle \tilde{\opA}y^{k+1}-\tilde{\opA}y^k,y^{k+1}-y^k\rangle,
\]
which can be verified by plugging in 
\[
y^{k+1}=\frac{1}{k+1}x^0+\frac{k}{k+1}(y^k-\tilde{\opA}y^k)-\tilde{\opA}y^{k+1}
\]
and performing basic (albeit somewhat tedious) calculations to check that all terms vanish.
By monotonicity of $\opA$, we have $V^{k+1}\le V^k$.
\end{frame}

\begin{frame}
\frametitle{Proof of Theorem~\ref{thm:appm}}

Since $V^k\le V^0=0$, we have
\begin{align*}
0&\ge V^k\\
&=k^2\|\tilde{\opA}y^k\|^2+k\langle \tilde{\opA}y^k,x^\star-x^0\rangle+k\langle \tilde{\opA}y^k,y^k-x^\star\rangle\\
&=
\frac{k^2}{2}\|\tilde{\opA}y^k\|^2
-\frac{1}{2}\|x^\star-x^0\|^2
+\frac{1}{2}\|k\tilde{\opA}y^k+x^\star-x^0\|^2
+
k\langle \tilde{\opA}y^k,y^k-x^\star\rangle\\
&\ge
\frac{k^2}{2}\|\tilde{\opA}y^k\|^2
-\frac{1}{2}\|x^\star-x^0\|^2,
\end{align*}
where the second equality follows from
% Lemma~\ref{lem:cosine} with $a=\tilde{\opA}y^k$ and $a-b=y^k-x^\star$
\[
k\langle \tilde{\opA}y^k,x^\star-x^0\rangle
=
\frac{1}{2}\|k\tilde{\opA}y^k+x^\star-x^0\|^2
-
\frac{k^2}{2}\| \tilde{\opA}y^k\|^2
-\frac{1}{2}\|x^\star-x^0\|^2
\]
 and the final inequality follows from monotonicity.\qed
 \end{frame}
 


\begin{frame}
\frametitle{When does an acceleration accelerate?}
In optimization (and more generally in applied mathematics and computer science), convergence rates are usually established in the worst case.
If an unaccelerated method actually converges at an $\mathcal{O}(1/k)$ rate, then an $\mathcal{O}(1/k^2)$ acceleration is a speedup.
However, if the observed convergence is already faster than $\mathcal{O}(1/k^2)$, then there is no guarantee of improvement. The acceleration may even slow down the convergence.


\vspace{0.2in}

In practice, an acceleration sometimes provides a speedup.
An ``acceleration'' should be tried it out with the expectation that it may improve or worsen the convergence.
\end{frame}


\end{document}
